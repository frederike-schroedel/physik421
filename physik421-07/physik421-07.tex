% Für Seitenformatierung

\documentclass[DIV=15]{scrartcl}

% Zeilenumbrüche

\parindent 0pt
\parskip 6pt

% Für deutsche Buchstaben und Synthax

\usepackage[ngerman]{babel}

% Für Auflistung mit speziellen Aufzählungszeichen

\usepackage{paralist}

% zB für \del, \dif und andere Mathebefehle

\usepackage{amsmath}
\usepackage{commath}
\usepackage{amssymb}

% Für Literatur/bibliography

%\usepackage[backend=biber , style=alphabetic , hyperref=true]{biblatex}

% Für \SIunit[]{} und \num in deutschem Stil

\usepackage[output-decimal-marker={,}]{siunitx}
\DeclareSIUnit\clight{\ensuremath{c}}

% Schriftart und encoding

\usepackage[utf8]{inputenc}
% Bitstream charter als default
\usepackage[charter, greekuppercase=italicized]{mathdesign}
% Lato, als sans default
\renewcommand{\sfdefault}{fla}

% Für \sfrac{}{}, also inline-frac

\usepackage{xfrac}

% Für Einbinden von pdf-Grafiken

\usepackage{graphicx}

% Umfließen von Bildern

\usepackage{floatflt}

% Für weitere Farben

\usepackage{color}

% Für Streichen von z.B. $\rightarrow$

\usepackage{centernot}

% Für Befehl \cancel{}

\usepackage{cancel}

% Für Links nach außen und innerhalb des Dokumentes

\usepackage{hyperref}

% Für Layout von Links

\hypersetup{
	citecolor=black,
	colorlinks=true,
	linkcolor=black,
	urlcolor=blue,
}

% Verschiedene Mathematik-Hilfen

\newcommand \e[1]{\cdot10^{#1}}
\newcommand\p{\partial}

\newcommand\half{\frac 12}
\newcommand\shalf{\sfrac12}

\newcommand\skp[2]{\left\langle#1,#2\right\rangle}
\newcommand\mw[1]{\left\langle#1\right\rangle}
\newcommand \eexp[1]{\mathrm{e}^{#1}}
\newcommand \dexp[1]{\exp\del{#1}}
\newcommand \dsin[1]{\sin\del{#1}}
\newcommand \dcos[1]{\cos\del{#1}}
\newcommand \dtan[1]{\tan\del{#1}}
\newcommand \darccos[1]{\arcos\del{#1}}
\newcommand \darcsin[1]{\arcsin\del{#1}}
\newcommand \darctan[1]{\arctan\del{#1}}

% Nabla und Kombinationen von Nabla

\renewcommand\div[1]{\skp{\nabla}{#1}}
\newcommand\rot{\nabla\times}
\newcommand\grad[1]{\nabla#1}
\newcommand\laplace{\triangle}
\newcommand\dalambert{\mathop{{}\Box}\nolimits}

%Für komplexe Zahlen

\newcommand \ii{\mathrm i}
\renewcommand{\Im}{\mathop{{}\mathrm{Im}}\nolimits}
\renewcommand{\Re}{\mathop{{}\mathrm{Re}}\nolimits}

%Für Bra-Ket-Notation

\newcommand\bra[1]{\left\langle#1\right|}
\newcommand\ket[1]{\left|#1\right\rangle}
\newcommand\braket[2]{\left\langle#1\left.\vphantom{#1 #2}\right|#2\right\rangle}
\newcommand\braopket[3]{\left\langle#1\left.\vphantom{#1 #2 #3}\right|#2\left.\vphantom{#1 #2 #3}\right|#3\right\rangle}


\newcounter{thezettel}
\setcounter{thezettel}{7}
\renewcommand\thesection{\arabic{thezettel}.\arabic{section}}

\newcommand{\ui}[1]{\int_{-\infty}^{\infty}\dif {#1}\;}
\newcommand\ccancel[2][black]{\renewcommand\CancelColor{\color{#1}}\cancel{#2}}


\title{physik421 - Übung \arabic{thezettel}}
\author{Lino Lemmer \\ \small{l2@uni-bonn.de} \and Frederike Schrödel \and Simon Schlepphorst\\ \small{s2@uni-bonn.de}}

\begin{document}
\maketitle

\section{Verschränkte Wellenfunktion}

\subsection{}

\begin{align*}
    \psi &= \frac 1{\sqrt2} \del{\psi_{1,\text{L}}\psi_{2,\text{L}}-\psi_{1,\text{R}}\psi_{2,\text{R}}} \\
         &= \frac1{\sqrt2}
    \del{\frac{\del{\psi_{1,x}-\ii\psi_{1,y}}\del{\psi_{2,x}+\ii\psi_{2,y}}}2 -
    \frac{\del{\psi_{1,x}+\ii\psi_{1,y}}\del{\psi_{2,x}-\ii\psi_{2,y}}}2} \\
    &= \frac1{\sqrt8}\del{\psi_{1,x}\psi_{2_x} + \ii\psi_{1,x}\psi_{2,y} - \ii\psi_{2,x}\psi_{1,y} + \psi_{1,y}\psi_{2,y} - \psi_{1,x}\psi_{2,x} +  \ii\psi_{1,x}\psi_{2,y} - \ii\psi_{1,y}\psi_{2,x} - \psi_{1,y}\psi_{2,y} } \\
    &= \frac\ii{\sqrt2}\del{\psi_{1,x}\psi_{2,y}-\psi_{1,y}\psi_{2,x}}
\end{align*}

\subsection{}

Ergibt die Messung für Photon 1 eine lineare Polarisation in $x$-Richtung
können wir
\begin{enumerate}[(i)]
    \item
        aus der Darstellung mit linearer Basis sagen, dass Photon 2 in $y$-Richtung polarisiert ist.
    \item
        aus der Darstellung in zirkularen Basis nichts heraus finden.
\end{enumerate}

\subsection{}

Nun ergibt eine Messung eine lineare Polarisation in $x$-Richtung für Photon 1 und eine linkshändig zirkulare Polarisation für Photon 2.
\begin{enumerate}[(i)]
    \item
        Wird die Messung an Photon 1 zuerst durchgeführt, kollabiert die Wellenfunktion zu
        \[
            \psi_I = \psi_{1,x}\psi_{2,y}.
        \]
    \item
        Wird die Messung an Photon 2 zuerst durchgeführt, kollabiert die Wellenfunktion zu
        \[
            \psi_{II} = \psi_{1,\text{L}}\psi_{2,\text{L}}.
        \]
\end{enumerate}

\subsection{}

\begin{enumerate}[(i)]
    \item
        In linearer Basis lassen sich die beiden Wellenfunktionen aus der letzten Teilaufgabe als
    \begin{align*}
        \psi_I &= \psi_{1,x}\psi_{2,y}
        \intertext{%
            bzw. als
        }
        \psi_{II} &= \frac{\del{\psi_{1,x}-\ii\psi_{1,y}}\del{\psi_{2,x}+\ii\psi_{2,y}}}2 \\
        &= \half\del{\psi_{1,x}\psi_{2_x} + \ii\psi_{1,x}\psi_{2,y} - \ii\psi_{2,x}\psi_{1,y} + \psi_{1,y}\psi_{2,y}}
    \end{align*}
    ausdrücken.
    \item
        In zirkularer Basis lassen sich die beiden Wellenfunktionen als
        \begin{align*}
            \psi_I &= \frac\ii{2} \del{\psi_{1,\text{R}}-\psi_{1,\text{L}}} \del{\psi_{2,\text{R}}-\psi_{2,\text{L}}} \\
                 &= \frac \ii2\del{\psi_{1,\text{R}}\psi_{2,\text{R}} - \psi_{1,\text{R}}\psi_{2,\text{L}} - \psi_{1,\text{L}}\psi_{2,\text{R}} + \psi_{1,\text{L}}\psi_{2,\text{L}}}
            \intertext{%
                bzw. als
            }
            \psi_{II} &= \psi_{1,\text{L}}\psi_{2,\text{L}}
        \end{align*}
        darstellen.
    \item
        Als Mischbasis wähle ich für $\psi_1$ die lineare und für $\psi_2$ die
        zirkulare Basis. So erhalte ich für die beiden Wellenfunktionen nach
        der Messung
        \begin{align*}
            \psi_I &= \frac\ii{\sqrt2} \psi_{1,x}\del{\psi_{2,\text{R}}-\psi_{2,\text{L}}}
            \intertext{%
                und
            }
            \psi_{II} &= \frac1{\sqrt2} \del{\psi_{1,x} - \ii\psi_{1,y}}\psi_{2,\text{L}}.
            \intertext{%
                Setzt man nun in $\psi_I$ die Transformation von $\psi_{2,\text{R}}$ ein erhält man
            }
            \psi_I &= \frac{\ii}{\sqrt{2}} \psi_{1,x}\del{\frac{1}{\sqrt2}\del{\psi_{2,x}-\ii\psi_{2,y}} - \psi_{2,\text{L}}} \\
                   &= \frac\ii2\psi_{1,x}\psi_{2,x} + \half \psi_{1,x}\psi_{2,y} - \frac\ii{\sqrt2}\psi_{1,x}\psi_{2,\text{L}} \\
                   &\overset{?}{=} \frac1{\sqrt2} \del{\psi_{1,x} - \ii\psi_{1,y}}\psi_{2,\text{L}} \\
            &= \psi_{II}
        \end{align*}
\end{enumerate}


\section{Hamilton-Operator für Teilchen in externen $\vec E$ und $\vec B$ Feldern}

\subsection{}
Es ist die Lagrange-Funktion für ein geladenes Teilchen in externen elektromagnetischen Feldern gegeben:
\[
    L = \frac 12 m(\dot\vec x)^2-q\del{V-\dot\vec x\vec A}
\]
Um den kanonisch konjugierten Impuls zu bestimmen nutze ich:
\[
    \vec P = \dpd{L}{\dot\vec x} = m\dot\vec x-q\vec A
\]

\subsection{}
Die dazu gehörige Hamilton-Funktion lautet:
\begin{align*}
    H &= \skp{\dot\vec q}{\vec P}-L \\
      &= \dot\vec x(m\dot\vec x+qA) -\frac 12 m(\dot\vec x)^2+q(V-\dot\vec xA) \\
      &= m\vec\dot x^2=\cancel{qA\vec\dot x}--\frac 12m\dot\vec x^2+qV-\cancel{q\dot\vec xA} \\
      &= \frac 12 m\vec\dot x^2+qV \\
      &= \frac{1}{2m}(m\vec\dot x)^2+qV \\
      &= \frac{1}{2m}(\vec p-q\vec A)+qV
\end{align*}

\subsection{}
Es soll gezeigt werden, dass die Hamilton-Funktion die Gesamt-Energie des
Systems beschreibt.Die Gesamt-Energie erhalten wir durch:
\[
    E_\text{ges} = T+V
\]
Diese Formel entspricht in den meisten Fällen bereits der Hamilton-Funktion.
In diesen Fall sollte man berücksichtigen, dass die kinetische Energie gegeben ist durch $T= \frac 12 \vec\dot x^2$,
die potentielle Energie lautet allerdings eigentlich nur $V=qV$, da das Magnetfeld nur eine Verschiebung verursacht.

\section{Wellenfunktion mit minimaler Unschärfe}

Betrachtet wird die in der Vorlesung hergeleitete Wellenfunktion mit minimaler Unschärfe
\[
	\psi_m\del{x}=\frac1{\del{2\pi\mw{\del{\Delta x}^2}}^{\frac14}}\ee^{-\frac{\del{x-\mw{x}}^2}{4\mw{\del{\Delta x}^2}}}
	\ee^{\frac{\ii x\mw{p_x}}{\hbar}}
\]

\subsection{Normierung}

Es soll gezeigt werden, dass obige Wellenfunktion normiert ist. Definiere $a\equiv\mw{\del{\Delta x}^2}$:
\begin{align*}
	\ui{x}\abs{\psi_m\del{x}}^2&=\ui{x}\psi_m^*\del{x}\psi_m\del{x}\\
	&=\ui{x}\frac1{\del{2\pi a}^{\frac12}}\ee^{-\frac{\del{x-\mw{x}}^2}{2a}}\\
	&=\frac1{\del{2\pi a}^{\frac12}}\ui{x}\ee^{-\frac1{2a}\del{x-\mw{x}}^2}\\
	\intertext{%
		Nach dem Hinweis von Aufgabenblatt 3, Aufgabe 3 folgt
	}
	&=\frac1{\del{2\pi a}^{\frac12}}\cdot\del{2\pi a}^{\frac12}\\
	&=1
\end{align*}
$\psi_m$ ist also normiert.

 Anm.: Die folgenden drei Aufgabenstellungen $3.2$ - $3.4$ sind leicht irritierend, eine Verwendung von $x'$ statt $x$ wäre evtl. sinnvoller gewesen.

\subsection{Erwartungswert von $x$}

Nun soll gezeigt werden, dass der Erwartungswert von $x$ gleich $\mw{x}$ in $\psi_m$ ist: 
\begin{align*}
	\mw{x}&=\ui{x}\psi_m^*\del{x}x\psi_m\del{x}\\
	&=\frac1{\del{2\pi a}^{\frac12}}\ui{x}x\ee^{-\frac{\del{x-\mw{x}}^2}{2a}}\\
	\intertext{%
		Substituiere $z=x-\mw{x}\iff x=z+\mw{x}\Rightarrow\dif x=\dif z$ 
	}
	&=\ui{z}\del{z+\mw{x}}\ee^{-\frac{z^2}{2a}}\\
	&=\underbrace{\frac1{\del{2\pi a}^{\frac12}}\ui{z}z\ee^{-\frac{z^2}{2a}}}_{=0}
	+\underbrace{\frac1{\del{2\pi a}^{\frac12}}\ui{z}\mw{x}\ee^{-\frac{z^2}{2a}}}_{=\mw{x}}\\
	&=\mw{x}
\end{align*}
Damit ist die Annahme gezeigt.

\subsection{Mittlere quadratische Abweichung von $x$ von seinem Mittelwert}

Nun soll gezeigt werden, dass $\mw{\del{\Delta x}^2}=\mw{x^2}-\mw{x}^2$ gleich $\mw{\del{\Delta x}^2}$ in $\psi_m$ ist:\\
Dazu berechne ich erst mal $\mw{x^2}$:
\begin{align*}
	\mw{x^2}&=\ui{x}\psi_m^*\del{x}x^2\psi_m\del{x}\\
	&=\frac1{\del{2\pi a}^{\frac12}}\ui{x}x^2\ee^{-\frac{\del{x-\mw{x}}^2}{2a}}\\
	\intertext{%
		Substituiere wieder $z=x-\mw{x}\iff x=z+\mw{x}\Rightarrow\dif x=\dif z$
	}
	&=\frac1{\del{2\pi a}^{\frac12}}\ui{z}\del{z+\mw{x}}^2\ee^{-\frac{z^2}{2a}}\\
	&=\frac1{\del{2\pi a}^{\frac12}}\ui{z}z^2\ee^{-\frac{z^2}{2a}}
	+\underbrace{\frac1{\del{2\pi a}^{\frac12}}\ui{z}2z\mw{x}\ee^{-\frac{z^2}{2a}}}_{=0}
	+\underbrace{\frac1{\del{2\pi a}^{\frac12}}\ui{z}\mw{x}^2\ee^{-\frac{z^2}{2a}}}_{=\mw{x}^2}\\
	\intertext{
		Mit partieller Integration von $\ui{z}z\cdot z\ee^{-\frac{z^2}{2a}}$ folgt
	}
	&=\underbrace{\left[az\ee^{-\frac{z^2}{2a}}\right]_{-\infty}^{\infty}}_{=0}
	+\underbrace{\frac1{\del{2\pi a}^{\frac12}}\ui{z}a\ee^{-\frac{z^2}{2a}}}_{=a}+\mw{x}^2
	\intertext{%
		Mit der oben stehenden Definition von $a$ folgt
	}
	&=\mw{\del{\Delta x}^2}+\mw{x}^2
\end{align*}
Und in der Tat gilt
\[
	\mw{\del{\Delta x}^2}=\mw{x^2}-\mw{x}^2=\mw{\del{\Delta x}^2}+\mw{x}^2-\mw{x}^2=\mw{\del{\Delta x}^2}
\]
Ich denke an dieser Stelle wird die Verwirrung auch recht klar.

\subsection{Erwartungswert von $p_x$}

Nun soll gezeigt werden, dass der Erwartungswert von $p_x$ gleich $\mw{p_x}$ in $\psi_m$ ist:
\begin{align*}
	\mw{p_x}&=\ui{x}\psi_m^*\del{x}p_x\psi_m\del{x}\\
	&=-\ui{x}\psi_m^*\del{x}\ii\hbar\dpd{}{x}\psi_m\del{x}\\
	&=-\frac{\ii\hbar}{\del{2\pi a}^{\frac12}}\ui{x}\ee^{-\frac{\del{x-\mw{x}}^2}{4a}}\ee^{-\frac{\ii x\mw{p_x}}{\hbar}}
	\dpd{}{x}\ee^{-\frac{\del{x-\mw{x}}^2}{4a}}\ee^{\frac{\ii x\mw{p_x}}{\hbar}}\\
	&=-\frac{\ii\hbar}{\del{2\pi a}^{\frac12}}\ui{x}\ee^{-\frac{\del{x-\mw{x}}^2}{4a}}\dpd{}{x}\ee^{-\frac{\del{x-\mw{x}}^2}{4a}}
	-\frac{\ii\hbar}{\del{2\pi a}^{\frac12}}\ui{x}\ee^{-\frac{\del{x-\mw{x}}^2}{2a}}\ee^{-\frac{\ii x\mw{p_x}}{\hbar}}
	\dpd{}{x}\ee^{\frac{\ii x\mw{p_x}}{\hbar}}\\
	&=-\frac{\ii\hbar}{\del{2\pi a}^{\frac12}}\ui{x}\frac1a\del{x-\mw{x}}\ee^{-\frac{\del{x-\mw{x}}^2}{2a}}
	-\underbrace{\frac{\ii\hbar}{\del{2\pi a}^{\frac12}}\ui{x}\frac{\ii\mw{p_x}}{\hbar}\ee^{-\frac{\del{x-\mw{x}}^2}{2a}}}_{=-\mw{p_x}}\\
	\intertext{%
		Substituiere mal wieder $z=x-\mw{x}\iff x=z+\mw{x}\Rightarrow\dif x=\dif z$
	}
	&=-\underbrace{\frac{\ii\hbar}{\del{2\pi a}^{\frac12}}\ui{z}\frac za\ee^{-\frac{z^2}{2a}}}_{=0}+\mw{p_x}\\
	&=\mw{p_x}
\end{align*}

\subsection{Mittlere quadratische Abweichung von $p_x$ und Unschärferelation}

Nun soll die mittlere quadratische Abweichung von $p_x$ berechnet werden. Dazu wieder zuerst $\mw{p_x^2}$:
\begin{align*}
	\mw{p_x^2}&=\ui{x}\psi_m^*\del{x}p_x^2\psi_m\del{x}\\
	&=-\ui{x}\psi_m^*\del{x}\hbar^2\dpd[2]{}{x}\psi_m\del{x}\\
	&=-\frac{\hbar^2}{\del{2\pi a}^{\frac12}}\ui{x}\ee^{-\frac{\del{x-\mw{x}}^2}{4a}}\ee^{-\frac{\ii x\mw{p_x}}{\hbar}}
	\dpd[2]{}{x}\ee^{-\frac{\del{x-\mw{x}}^2}{4a}}\ee^{\frac{\ii x\mw{p_x}}{\hbar}}\\
	&=-\frac{\hbar^2}{\del{2\pi a}^{\frac12}}\ui{x}\ee^{-\frac{\del{x-\mw{x}}^2}{4a}}\dpd[2]{}{x}\ee^{-\frac{\del{x-\mw{x}}^2}{4a}}
	-\frac{\hbar^2}{\del{2\pi a}^{\frac12}}\ui{x}\ee^{-\frac{\del{x-\mw{x}}^2}{2a}}\ee^{-\frac{\ii x\mw{p_x}}{\hbar}}
	\dpd[2]{}{x}\ee^{\frac{\ii x\mw{p_x}}{\hbar}}\\
	&=-\frac{\hbar^2}{\del{2\pi a}^{\frac12}}\ui{x}\ee^{-\frac{\del{x-\mw{x}}^2}{4a}}
	\frac{-1}{2a}\dpd{}x\del{x-\mw{x}}\ee^{-\frac{\del{x-\mw{x}}^2}{4a}}
	+\underbrace{\frac{\hbar^2}{\del{2\pi a}^{\frac12}}\ui{x}\frac{\mw{p_x}^2}{\hbar^2}\ee^{-\frac{\del{x-\mw{x}}^2}{2a}}}_{=\mw{p_x}^2}\\
	&=\mw{p_x}^2-\frac{\hbar^2}{\del{2\pi a}^{\frac12}}\ui{x}\ee^{-\frac{\del{x-\mw{x}}^2}{4a}}
	\frac1{4a^2}\del{-2a+\del{x^2-x\mw{x}}-\del{x\mw{x}-\mw{x}^2}}\ee^{-\frac{\del{x-\mw{x}}^2}{4a}}\\
	&=\mw{p_x}^2-\frac{\hbar^2}{4a^2\del{2\pi a}^{\frac12}}\ui{x}\del{-2a+x^2-2x\mw{x}+\mw{x}^2}\ee^{-\frac{\del{x-\mw{x}}^2}{2a}}\\
	&=\mw{p_x}^2-\frac{\hbar^2}{4a^2\del{2\pi a}^{\frac12}}\ui{x}\del{-2a+\del{x-\mw{x}}^2}\ee^{-\frac{\del{x-\mw{x}}^2}{2a}}\\
	\intertext{%
		Substituiere mal wieder $z=x-\mw{x}\iff x=z+\mw{x}\Rightarrow\dif x=\dif z$
	}
	&=\mw{p_x}^2-\frac{\hbar^2}{4a^2\del{2\pi a}^{\frac12}}\ui{z}\del{-2a+z^2}\ee^{-\frac{z^2}{2a}}\\
	&=\mw{p_x}^2-\frac{\hbar^2}{4a^2}\del{-2a\underbrace{\frac1{\del{2\pi a}^{\frac12}}\ui{z}\ee^{-\frac{z^2}{2a}}}_{=1}
	+\underbrace{\frac1{\del{2\pi a}^{\frac12}}\ui{z}z^2\ee^{-\frac{z^2}{2a}}}_{=a}}\\
	&==\mw{p_x}^2+\frac{\hbar^2}{4a}
\end{align*}
Daraus folgt mit der Definition von $a$:
\[
	\mw{\del{\Delta p_x}^2}=\mw{p_x^2}-\mw{p_x}^2=\frac{\hbar^2}{4\mw{\del{\Delta x}^2}}=
\]
Damit ist
\[
	\mw{\del{\Delta x}^2}\cdot\mw{\del{\Delta p_x}^2}=\frac{\hbar^2}4
\]
Nach der Aufgabenstellung sollte $\frac{\hbar^4}4$ herauskommen, aber dann würde sich nicht die Unschärferelation ergeben, daher bin ich mir sicher, dass mein Ergebnis richtig ist.

\end{document}
