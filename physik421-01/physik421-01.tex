% Für Seitenformatierung

\documentclass[DIV=15]{scrartcl}

% Zeilenumbrüche

\parindent 0pt
\parskip 6pt

% Für deutsche Buchstaben und Synthax

\usepackage[ngerman]{babel}

% Für Auflistung mit speziellen Aufzählungszeichen

\usepackage{paralist}

% zB für \del, \dif und andere Mathebefehle

\usepackage{amsmath}
\usepackage{commath}
\usepackage{amssymb}

% Für Literatur/bibliography

%\usepackage[backend=biber , style=alphabetic , hyperref=true]{biblatex}

% Für \SIunit[]{} und \num in deutschem Stil

\usepackage[output-decimal-marker={,}]{siunitx}
\DeclareSIUnit\clight{\ensuremath{c}}

% Schriftart und encoding

\usepackage[utf8]{inputenc}
% Bitstream charter als default
\usepackage[charter, greekuppercase=italicized]{mathdesign}
% Lato, als sans default
\renewcommand{\sfdefault}{fla}

% Für \sfrac{}{}, also inline-frac

\usepackage{xfrac}

% Für Einbinden von pdf-Grafiken

\usepackage{graphicx}

% Umfließen von Bildern

\usepackage{floatflt}

% Für weitere Farben

\usepackage{color}

% Für Streichen von z.B. $\rightarrow$

\usepackage{centernot}

% Für Befehl \cancel{}

\usepackage{cancel}

% Für Links nach außen und innerhalb des Dokumentes

\usepackage{hyperref}

% Für Layout von Links

\hypersetup{
	citecolor=black,
	colorlinks=true,
	linkcolor=black,
	urlcolor=blue,
}

% Verschiedene Mathematik-Hilfen

\newcommand \e[1]{\cdot10^{#1}}
\newcommand\p{\partial}

\newcommand\half{\frac 12}
\newcommand\shalf{\sfrac12}

\newcommand\skp[2]{\left\langle#1,#2\right\rangle}
\newcommand\mw[1]{\left\langle#1\right\rangle}
\newcommand \eexp[1]{\mathrm{e}^{#1}}
\newcommand \dexp[1]{\exp\del{#1}}
\newcommand \dsin[1]{\sin\del{#1}}
\newcommand \dcos[1]{\cos\del{#1}}
\newcommand \dtan[1]{\tan\del{#1}}
\newcommand \darccos[1]{\arcos\del{#1}}
\newcommand \darcsin[1]{\arcsin\del{#1}}
\newcommand \darctan[1]{\arctan\del{#1}}

% Nabla und Kombinationen von Nabla

\renewcommand\div[1]{\skp{\nabla}{#1}}
\newcommand\rot{\nabla\times}
\newcommand\grad[1]{\nabla#1}
\newcommand\laplace{\triangle}
\newcommand\dalambert{\mathop{{}\Box}\nolimits}

%Für komplexe Zahlen

\newcommand \ii{\mathrm i}
\renewcommand{\Im}{\mathop{{}\mathrm{Im}}\nolimits}
\renewcommand{\Re}{\mathop{{}\mathrm{Re}}\nolimits}

%Für Bra-Ket-Notation

\newcommand\bra[1]{\left\langle#1\right|}
\newcommand\ket[1]{\left|#1\right\rangle}
\newcommand\braket[2]{\left\langle#1\left.\vphantom{#1 #2}\right|#2\right\rangle}
\newcommand\braopket[3]{\left\langle#1\left.\vphantom{#1 #2 #3}\right|#2\left.\vphantom{#1 #2 #3}\right|#3\right\rangle}


\newcounter{thezettel}
\setcounter{thezettel}{1}
\renewcommand\thesection{\arabic{thezettel}.\arabic{section}}

\title{physik421 - Übung \arabic{thezettel}}
\author{Lino Lemmer \and Frederike Schrödel \\ \small{l2@uni-bonn.de}}

\begin{document}
\maketitle

\section{Komplexe Zahlen}

Gegeben sind $z_1 = a_1 + \ii b_1$ und $z_2 = a_2 + \ii b_2$.

\subsection{Summe und Produkt}

\begin{align*}
    z_1 + z_2 &= a_1 + a_2 + \ii\del{b_1 + b_2} \\
    z_1 \cdot z_2 &= a_1a_2 - b_1b_2 + \ii\del{a_1b_2+a_2b_1}
\end{align*}

\subsection{Absolutbetrag}

\begin{align*}
    \abs{z_1} &= \sqrt{\del{a_1 + \ii b_1}\del{a_1 - \ii b_1}} \\
              &= \sqrt{a_1^2 + b_1^2}
\end{align*}

\subsection{Konjungieren eines Produktes}

\begin{align*}
    \del{z_1z_2}^* &= \del{a_1a_2 - b_1b_2 + \ii\del{a_1b_2 + a_2b_1}}^* \\
                   &= a_1a_2 - b_1b_2 - \ii\del{a_1b_2 + a_2b_1} \\
                   &= \del{a_1 - \ii b_1}\del{a_2 - \ii b_2} \\
                   &= z_1^* z_2^*
\end{align*}

\subsection{Polare Darstellung}

\begin{align*}
    r_j &= \abs{z_j} \\
        &= \sqrt{a_j^2 + b_j^2} \\
    \phi_j &= \darccos{\frac{b_j}{a_j}}
\end{align*}

\subsection{Komplex Konjungierte}

\begin{align*}
    z_j &= r_j\eexp{\ii\phi_j} \\
    z_j^* &= r_j\eexp{-\ii\phi_j}
\end{align*}

\subsection{Produkt und Quotient}

\begin{align*}
    z_1z_2 &= r_1r_2\eexp{\ii\del{\phi_1+\phi_2}} \\
    \frac{z_1}{z_2} &= \frac{r_1}{r_2} \eexp{\ii\del{\phi_1-\phi_2}}
\end{align*}

\subsection{Betrag einer Summe}

\begin{align*}
    \abs{z_1+z_2} &= \sqrt{\del{z_1+z_2}\del{z_1+z_2}^*} \\
                  &= \sqrt{ \del{a_1+a_2 + \ii\del{b_1+b_2}}\del{a_1+a_2 - \ii\del{b_1+b_2}}} \\
                  &= \sqrt{\del{a_1+a_2}^2 + \del{b_1+b_2}^2}
\end{align*}

\subsection{Eulersche Summe}

Die Taylor-Entwicklungen sind
\begin{align*}
    \cos x &= \frac{x^0}{0!} - \frac{x^2}{2!} + \frac{x^4}{4!} + \dots \\
           &= 1 - \frac{x^2}{2} + \frac{x^4}{24} + \dots\\
    \sin x &= \frac{x^1}{1!} - \frac{x^3}{3!} + \frac{x^5}{5!} + \dots \\
           &= x - \frac{x^3}{6} + \frac{x^5}{120} + \dots \\
    \eexp{\ii x} &= \frac{\del{\ii x}^0}{0!} + \frac{\del{\ii x}^1}{1!} +
    \frac{\del{\ii x}^2}{2!} + \frac{\del{\ii x}^3}{3!} + 
    \frac{\del{\ii x}^4}{4!} +\frac{\del{\ii x}^5}{5!} + \dots \\
    &= 1 + \ii x - \frac{x^2}{2} - \ii\frac{x^3}{6} + \frac{x^4}{24} +
    \ii\frac{x^5}{120} + \dots.
    \intertext{%
        Man sieht sofort, dass gilt
    }
    \eexp{\ii x} &= \cos x + \ii\sin x.
\end{align*}

\section{Interferenz ebener Wellen}

Gegben sind zwei ebene Wellen der Form:
\begin{align*}
	\Psi_1=\vec{A}_1\ee^{\ii\del{\omega t-\vec{k}_1\vec{x}}}\\
	\Psi_2=\vec{A}_2\ee^{\ii\del{\omega t-\vec{k}_2\vec{x}}}
\end{align*}
Es soll angenommen werden, diese seien kohärent.
\subsection{Intensität}
Es soll die Intensität $I$ der beiden Wellen berechnet werden:
\begin{align*}
	I&=\abs{\Psi_1+\Psi_2}^2\\
	&=\abs{\vec{A}_1\ee^{\ii\del{\omega t-\vec{k}_1\vec{x}}}+\vec{A}_2\ee^{\ii\del{\omega t-\vec{k}_2\vec{x}}}}^2\\
	&=\del{\vec{A}_1\ee^{\ii\del{\omega t-\vec{k}_1\vec{x}}}+\vec{A}_2\ee^{\ii\del{\omega t-\vec{k}_2\vec{x}}}}
	\del{\vec{A}_1\ee^{-\ii\del{\omega t-\vec{k}_1\vec{x}}}+\vec{A}_2\ee^{-\ii\del{\omega t-\vec{k}_2\vec{x}}}}\\
	&=\vec{A}_1^2+\vec{A}_2^2+\vec{A}_1\vec{A}_2\del{\ee^{\ii\vec{x}\del{\vec{k}_1-\vec{k}_2}}+\ee^{\ii\vec{x}\del{\vec{k}_2-\vec{k}_1}}}\\
	&=\vec{A}_1^2+\vec{A}_2^2+\vec{A}_1\vec{A}_2\del{\ee^{\ii\vec{x}\del{\vec{k}_1-\vec{k}_2}}+\ee^{-\ii\vec{x}\del{\vec{k}_1-\vec{k}_2}}}\\
	&=\vec{A}_1^2+\vec{A}_2^2+2\vec{A}_1\vec{A}_2\cos\del{\vec{x}\del{\vec{k}_1-\vec{k}_2}}\\
\end{align*}

\subsection{Inkohärente Intensität}
Nun soll die Intensität betrachtet werden, wenn die beiden Wellen nicht kohärent sind. In diesem Fall gilt für die Intensität:
\begin{align*}
	I=\abs{\Psi_1^2+\Psi_2^2}
\end{align*}

\subsection{Minima und Maxima der Intensität}
Jetzt soll errechnet werden, für welche $\vec{k}_j$ die Intensität minimal/maximal wird. Aufgrund der Cosinus-Abhängigkeit der Intensität ist diese minimal für
\begin{align*}
	\cos\del{\vec{x}\del{\vec{k}_1-\vec{k}_2}}=-1\iff\vec{x}\del{\vec{k}_1-\vec{k}_2}=\del{2n+1}\pi&& n\in \mathbb N
\end{align*}
und maximal für
\begin{align*}
	\cos\del{\vec{x}\del{\vec{k}_1-\vec{k}_2}}=1\iff\vec{x}\del{\vec{k}_1-\vec{k}_2}=2n\pi&& n\in\mathbb N
\end{align*}
Der demnach einfachste Fall für ein Maximum wäre bei $\vec{k}_1=\vec{k}_2$

\subsection{Aufheben der beiden Wellen}

Nun soll gezeigt werden unter welchen Bedingungen sich die beiden Wellen exakt aufheben würden. Aus 2.3 ist bekannt, dass das Minimum der Intensität erreicht wird, wenn $\vec{x}\del{\vec{k}_1-\vec{k}_2}=\del{2n+1}\pi$. In diesem Fall ist die Intensität:
\begin{align*}
	I=\vec{A}_1^2+\vec{A}_2^2-2\vec{A}_1\vec{A}_2
\end{align*}
Dabei heben sich die Wellen auf wenn $I=0$ ist:
\begin{align*}
	\vec{A}_1^2+\vec{A}_2^2-2\vec{A}_1\vec{A}_2=0
\end{align*}
Betrachtet man die Beträge $\abs{\vec{A}_1}$ und $\abs{\vec{A}_2}$ so folgt:
\begin{align*}
	\abs{\vec{A}_1}=\abs{\vec{A}_2}
\end{align*}
Die beiden Bedingungen dafür, dass sich die beiden Wellen aufheben sind also, dass die Intensität minimal wird und die Amplituden denselben Betrag haben.


\section{Beugung am Einzelspalt}

\subsection{Gangunterschied benachbarter Bündel}

\begin{figure}[htbp]
    \centering
    \begin{tikzpicture}
        \draw[thick,<->]
        (0,0) -- node[left]{$\frac{d}{2N}$} (0,2)
        ;
        \draw[thick]
        (.2,2) -- (2,2) -- (6,1)
        (2,1) -- (6,0)
        (.2,0) -- (2,0) -- (6,-1)
        ;
        \draw[dashed]
        (2,2.5) node[above] {Einzelspalt} -- (2,-.5)
        ;
        \draw[dashdotted]
        (2,2) -- (6,2)
        (2,0) -- (intersection cs: first line={(2,0)--(3,4)}, second line={(2,2)--(6,1)})
        ;
        \draw[bend right]
        (5,1.25) to (5.2,2)
        (2.2,0.825) to (2,.9)
        ;
        \draw
        (2.05,.8) -- (3,.6) node[below]{$\alpha$} -- (4,1.8)
        ;
        \node () at (2.2,1.75) {$\Delta$};
    \end{tikzpicture}
    \caption{%
        Zwei benachbarte Bündel aus dem Lichtstrahl.
    }
    \label{fig:Einzelspalt}
\end{figure}

Der Gangunterschied $\Delta$ zwischen zwei benachbarten Bündeln ist, wie aus
Abbildung~\ref{fig:Einzelspalt} ersichtlich $\Delta = \frac{d}{2N}\sin\alpha$.

\subsection{Intensitätsminimum}

Für ein Intensitätsminimum müssen die beiden äußeren Strahlen einen Gangunterschied von
\begin{align*}
    \Delta &= k\lambda,
    \intertext{%
        mit $k\in\mathbb{N}$ haben, dann gibt es immer zwei Strahlen mit Gangunterschied $\frac{\lambda}{2}$. Beugungsminima treten daher unter
    }
    \alpha &= \darcsin{\frac{2Nk\lambda}{d}}
\end{align*}

\subsection{Intensitätsmaximum}

Für ein Intensitätsmaximum muss gelten
\begin{align*}
    \Delta &=  \del{k+\half}\lambda,
    \intertext{%
        mit $k\in\mathbb{N}$. Beugungsminima treten daher unter
    }
    \alpha &= \darcsin{\frac{\del{k+\half}2N\lambda}{d}}
\end{align*}

\subsection{Bedingung an Spaltbreite}

Damit Beugungsmuster beobachtet werden können muss $d>\lambda$ gelten.

\end{document}
