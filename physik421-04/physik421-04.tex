% Für Seitenformatierung

\documentclass[DIV=15]{scrartcl}

% Zeilenumbrüche

\parindent 0pt
\parskip 6pt

% Für deutsche Buchstaben und Synthax

\usepackage[ngerman]{babel}

% Für Auflistung mit speziellen Aufzählungszeichen

\usepackage{paralist}

% zB für \del, \dif und andere Mathebefehle

\usepackage{amsmath}
\usepackage{commath}
\usepackage{amssymb}

% Für Literatur/bibliography

%\usepackage[backend=biber , style=alphabetic , hyperref=true]{biblatex}

% Für \SIunit[]{} und \num in deutschem Stil

\usepackage[output-decimal-marker={,}]{siunitx}
\DeclareSIUnit\clight{\ensuremath{c}}

% Schriftart und encoding

\usepackage[utf8]{inputenc}
% Bitstream charter als default
\usepackage[charter, greekuppercase=italicized]{mathdesign}
% Lato, als sans default
\renewcommand{\sfdefault}{fla}

% Für \sfrac{}{}, also inline-frac

\usepackage{xfrac}

% Für Einbinden von pdf-Grafiken

\usepackage{graphicx}

% Umfließen von Bildern

\usepackage{floatflt}

% Für weitere Farben

\usepackage{color}

% Für Streichen von z.B. $\rightarrow$

\usepackage{centernot}

% Für Befehl \cancel{}

\usepackage{cancel}

% Für Links nach außen und innerhalb des Dokumentes

\usepackage{hyperref}

% Für Layout von Links

\hypersetup{
	citecolor=black,
	colorlinks=true,
	linkcolor=black,
	urlcolor=blue,
}

% Verschiedene Mathematik-Hilfen

\newcommand \e[1]{\cdot10^{#1}}
\newcommand\p{\partial}

\newcommand\half{\frac 12}
\newcommand\shalf{\sfrac12}

\newcommand\skp[2]{\left\langle#1,#2\right\rangle}
\newcommand\mw[1]{\left\langle#1\right\rangle}
\newcommand \eexp[1]{\mathrm{e}^{#1}}
\newcommand \dexp[1]{\exp\del{#1}}
\newcommand \dsin[1]{\sin\del{#1}}
\newcommand \dcos[1]{\cos\del{#1}}
\newcommand \dtan[1]{\tan\del{#1}}
\newcommand \darccos[1]{\arcos\del{#1}}
\newcommand \darcsin[1]{\arcsin\del{#1}}
\newcommand \darctan[1]{\arctan\del{#1}}

% Nabla und Kombinationen von Nabla

\renewcommand\div[1]{\skp{\nabla}{#1}}
\newcommand\rot{\nabla\times}
\newcommand\grad[1]{\nabla#1}
\newcommand\laplace{\triangle}
\newcommand\dalambert{\mathop{{}\Box}\nolimits}

%Für komplexe Zahlen

\newcommand \ii{\mathrm i}
\renewcommand{\Im}{\mathop{{}\mathrm{Im}}\nolimits}
\renewcommand{\Re}{\mathop{{}\mathrm{Re}}\nolimits}

%Für Bra-Ket-Notation

\newcommand\bra[1]{\left\langle#1\right|}
\newcommand\ket[1]{\left|#1\right\rangle}
\newcommand\braket[2]{\left\langle#1\left.\vphantom{#1 #2}\right|#2\right\rangle}
\newcommand\braopket[3]{\left\langle#1\left.\vphantom{#1 #2 #3}\right|#2\left.\vphantom{#1 #2 #3}\right|#3\right\rangle}


\newcounter{thezettel}
\setcounter{thezettel}{4}
\renewcommand\thesection{\arabic{thezettel}.\arabic{section}}


\title{physik421 - Übung \arabic{thezettel}}
\author{Lino Lemmer \\ \small{l2@uni-bonn.de} \and Frederike Schrödel \and Simon Schlepphorst\\ \small{s2@uni-bonn.de}}

\begin{document}
\maketitle

\section{Hermitesche Polynome}

Die Hermitischen Polynome haben die Integraldarstellung
\[
    H_n(x) = \frac1{\sqrt\piup}2^n\int_{-\infty}^\infty\!\dif y\, (x + \ii y)^n \eexp{-y^2}
\]

\subsection{Erste drei Polynome}

Mit der Integraldarstellung sollen die ersten drei Polynome berechnet werden.
\begin{align*}
    H_0(x) &= \frac1{\sqrt\piup} 2^0 \int_{-\infty}^\infty\!\dif y\, (x + \ii y)^0 \eexp{-y^2} \\
           &= \frac1{\sqrt\piup} \int_{-\infty}^\infty\!\dif y\, \eexp{-y^2} \\
           &= \frac1{\sqrt\piup} \sqrt\piup \\
           &= 1 \\
    H_1(x) &= \frac1{\sqrt\piup} 2^1 \int_{-\infty}^\infty\!\dif y\, (x + \ii y)^1 \eexp{-y^2} \\
           &= \frac{2}{\sqrt\piup}\del{x\int_{-\infty}^\infty\!\dif y\, \eexp{-y^2} + \ii\int_{-\infty}^\infty\!\dif y\,\underbrace{y}_{\text{unger.}} \underbrace{\eexp{-y^2}}_{\text{ger.}}} \\
           &= \frac{2x}{\sqrt\piup}\sqrt\piup + 0 \\
           &= 2x \\
    H_2(x) &= \frac1{\sqrt\piup} 2^2 \int_{-\infty}^\infty\!\dif y\, (x + \ii y)^2 \eexp{-y^2} \\
           &= \frac4{\sqrt\piup} \del{x^2\int_{-\infty}^\infty\!\dif y\, \eexp{-y^2} + \underbrace{2\ii x \int_{-\infty}^\infty\!\dif y\,y\eexp{-y^2}}_{=0} - \int_{-\infty}^\infty\!\dif y\, y^2\eexp{-y^2}}
    \intertext{%
        Hier wenden wir einen kleinen Trick an. Wir benutzen, dass
    $
    y^2\eexp{-y^2} = \left.-\dod{}{c}\eexp{-cy^2}\right|_{c=1}
    $
        gilt. Da diese Funktion stetig und genügend häufig differenzierbar ist, können wir Ableitung und Integration vertauschen:
    }
    &= 4x^2 + \frac4{\sqrt\piup}\left.\dod{}{c}\int_{-\infty}^\infty\!\dif y\, \eexp{-cy^2}\right|_{c=1} \\
    &= 4x^2 + \frac4{\sqrt\piup}\left.\dod{}{c}\sqrt{\frac\piup c}\right|_{c=1} \\
    &= 4x^2 - 4\cdot\left.\frac1{2\sqrt{c}^3}\right|_{c=1} \\
    &= 4x^2 - 2
\end{align*}

\subsection{Erzeugende Funktion}

Zu zeigen ist, dass 
\[
    \sum_{n=0}^\infty \frac{t^n}{n!}H_n(x) = \dexp{2tx -t^2}
\]
gilt.
\begin{align*}
    \sum_{n=0}^\infty \frac{t^n}{n!}H_n(x) &=
    \frac1{\sqrt\pi}\int_{-\infty}^{\infty}\!\dif y\, \sum_{n=0}^\infty
    \frac{(2t)^n}{n!} (x - \ii y)^n \dexp{-y^2} \\
    &= \frac1{\sqrt\pi}\int_{-\infty}^{\infty}\!\dif y\, \dexp{2tx - 2t\ii y - y^2} \\
    &= \frac1{\sqrt\pi}\int_{-\infty}^{\infty}\!\dif y\, \dexp{2tx -t^2 + t^2 - 2t\ii y - y^2} \\
    &= \frac1{\sqrt\pi}\int_{-\infty}^{\infty}\!\dif y\, \dexp{2tx -t^2 - (\ii t+y)^2} \\
    &= \frac{\dexp{2tx -t^2}}{\sqrt\pi}\int_{-\infty}^{\infty}\!\dif y\, \dexp{-(\ii t+y)^2} 
    \intertext{%
        Mit $z = \ii t + y$ und $\dif z = \dif y$ erhalten wir
    }
    &= \frac{\dexp{2tx -t^2}}{\sqrt\pi} \int_{-\infty}^{\infty}\!\dif z\, \dexp{-z^2} \\
    &= \dexp{2tx -t^2}.
\end{align*}

\section{Eigenzustände des harmonischen Oszillators}


\section{Klassisch verbotener Bereich}


\section{Harmonischer Oszillator mit zusätzlichem Potenzial}


\end{document}
