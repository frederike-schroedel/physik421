% Für Seitenformatierung

\documentclass[DIV=15]{scrartcl}

% Zeilenumbrüche

\parindent 0pt
\parskip 6pt

% Für deutsche Buchstaben und Synthax

\usepackage[ngerman]{babel}

% Für Auflistung mit speziellen Aufzählungszeichen

\usepackage{paralist}

% zB für \del, \dif und andere Mathebefehle

\usepackage{amsmath}
\usepackage{commath}
\usepackage{amssymb}

% Für Literatur/bibliography

%\usepackage[backend=biber , style=alphabetic , hyperref=true]{biblatex}

% Für \SIunit[]{} und \num in deutschem Stil

\usepackage[output-decimal-marker={,}]{siunitx}
\DeclareSIUnit\clight{\ensuremath{c}}

% Schriftart und encoding

\usepackage[utf8]{inputenc}
% Bitstream charter als default
\usepackage[charter, greekuppercase=italicized]{mathdesign}
% Lato, als sans default
\renewcommand{\sfdefault}{fla}

% Für \sfrac{}{}, also inline-frac

\usepackage{xfrac}

% Für Einbinden von pdf-Grafiken

\usepackage{graphicx}

% Umfließen von Bildern

\usepackage{floatflt}

% Für weitere Farben

\usepackage{color}

% Für Streichen von z.B. $\rightarrow$

\usepackage{centernot}

% Für Befehl \cancel{}

\usepackage{cancel}

% Für Links nach außen und innerhalb des Dokumentes

\usepackage{hyperref}

% Für Layout von Links

\hypersetup{
	citecolor=black,
	colorlinks=true,
	linkcolor=black,
	urlcolor=blue,
}

% Verschiedene Mathematik-Hilfen

\newcommand \e[1]{\cdot10^{#1}}
\newcommand\p{\partial}

\newcommand\half{\frac 12}
\newcommand\shalf{\sfrac12}

\newcommand\skp[2]{\left\langle#1,#2\right\rangle}
\newcommand\mw[1]{\left\langle#1\right\rangle}
\newcommand \eexp[1]{\mathrm{e}^{#1}}
\newcommand \dexp[1]{\exp\del{#1}}
\newcommand \dsin[1]{\sin\del{#1}}
\newcommand \dcos[1]{\cos\del{#1}}
\newcommand \dtan[1]{\tan\del{#1}}
\newcommand \darccos[1]{\arcos\del{#1}}
\newcommand \darcsin[1]{\arcsin\del{#1}}
\newcommand \darctan[1]{\arctan\del{#1}}

% Nabla und Kombinationen von Nabla

\renewcommand\div[1]{\skp{\nabla}{#1}}
\newcommand\rot{\nabla\times}
\newcommand\grad[1]{\nabla#1}
\newcommand\laplace{\triangle}
\newcommand\dalambert{\mathop{{}\Box}\nolimits}

%Für komplexe Zahlen

\newcommand \ii{\mathrm i}
\renewcommand{\Im}{\mathop{{}\mathrm{Im}}\nolimits}
\renewcommand{\Re}{\mathop{{}\mathrm{Re}}\nolimits}

%Für Bra-Ket-Notation

\newcommand\bra[1]{\left\langle#1\right|}
\newcommand\ket[1]{\left|#1\right\rangle}
\newcommand\braket[2]{\left\langle#1\left.\vphantom{#1 #2}\right|#2\right\rangle}
\newcommand\braopket[3]{\left\langle#1\left.\vphantom{#1 #2 #3}\right|#2\left.\vphantom{#1 #2 #3}\right|#3\right\rangle}



\newcounter{thezettel}
\setcounter{thezettel}{6}
\renewcommand\thesection{\arabic{thezettel}.\arabic{section}}

\newcommand\ccancel[2][black]{\renewcommand\CancelColor{\color{#1}}\cancel{#2}}


\title{physik421 - Übung \arabic{thezettel}}
\author{Lino Lemmer \\ \small{l2@uni-bonn.de} \and Frederike Schrödel \and Simon Schlepphorst\\ \small{s2@uni-bonn.de}}

\begin{document}
\maketitle

In diesen Aufgaben lassen wir die Hütchen auf den Operatoren $\hat A$, $\hat B$
und $\hat C$ weg, da es sich hier bei immer um Operatoren handelt.

\section{Kommutatoren von linearen Operatoren}

\subsection{}

\begin{align*}
    \sbr{A,BC} &= ABC - BCA \\
               &= ABC - BAC + BAC -BCA \\
               &= \del{AB - BA}C + B\del{AC - CA} \\
               &= \sbr{A,B}C + B\sbr{A,C}
\end{align*}

\subsection{}

\begin{align*}
    \sbr{AB,C} &= ABC - CAB \\
               &= ABC - ACB + ACB - CAB \\
               &= A\del{BC-CB} + \del{AC-CA}B \\
               &= A\sbr{B,C} + \sbr{A,C}B
\end{align*}

\subsection{}

\begin{align*}
    & \sbr{A,\sbr{B,C}} + \sbr{B,\sbr{C,A}} + \sbr{C\sbr{A,B}}\\
    &= A\del{BC-CB} - \del{BC-CB}A \\
    &+ B\del{CA-AC} - \del{CA-AC}B \\
    &+ C\del{AB-BA} - \del{AB-BA}C \\
    &= \ccancel[red]{ABC} - \ccancel[orange]{ACB} - \ccancel[yellow]{BCA} + \ccancel[green]{CBA} \\
    &+ \ccancel[yellow]{BCA} - \ccancel[blue]{BAC} - \ccancel[gray]{CAB} + \ccancel[orange]{ACB} \\
    &+ \ccancel[gray]{CAB} - \ccancel[green]{CBA} - \ccancel[red]{ABC} + \ccancel[blue]{BAC} \\
    &= 0
\end{align*}

\subsection{}

Aus $\sbr{A,C}=0$ und $\sbr{B,C}=0$ folgt nicht notwendigerweise $\sbr{A,B}=0$. 
Sei $C=1$, so folgt $\sbr{A,C}=0$ und $\sbr{B,C}=0$, nicht jedoch zwangsweise $\sbr{A,B}=0$. Beispiel: $A=\hat x$, $B=\hat p_x$.

\section{Kommutatoren von Operatorfunktionen}

$A$ und $B$ kommutieren mit $C$ aber nicht miteinander: $\sbr{A,B}=C$. Operatorfunktionen sind als Potenzreihen definiert.

\subsection{$\sbr{A,f(B)}=Cf'(B)$}

\begin{align*}
    f(B) &= \sum_{n=0} \beta_nB^n
    \intertext{Behauptung:}
    \sbr{A,B^n} &= nCB^{n-1}
    \intertext{Beweis durch vollständige Induktion:\newline $n=1$:}
    \sbr{A,B} &= C \\
              &= 1CB^0
    \intertext{$n=2$:}
    \sbr{A,B^2} &= ABB - BBA \\
                &= ABB - BAB + BAB - BBA \\
                &= \sbr{A,B}B + B\sbr{A,B} \\
                &= CB + BC \\
                &= 2CB
    \intertext{$n \to n+1$:}
    \sbr{A,B^{n+1}} &= AB^{n+1} - B^{n+1}A \\
                    &= AB^nB - B^nAB + B^nAB - B^nBA \\
                    &= \sbr{A,B^n}B + B^n\sbr{A,B} \\
                    &= nCB^{n-1}B + B^nC \\
                    &= nCB^n + CB^n \\
                    &= (n+1)CB^n
    \intertext{Behauptung durch vollständige Induktion Bewiesen. Damit folgt}
    \sbr{A,f(B)} &= \sum_{n=0} \beta_n \sbr{A,B^n} \\
                 &= C\sum_{n=0} \beta_n nB^{n-1} \\
                 &= C\dod{}{B}\sum_{n=0} \beta_n B^n \\
                 &= C\dod{}{B} f(B) \\
                 &= Cf'(B).
\end{align*}

\subsection{$\sbr{g(A),B}=Cg'(A)$}

\begin{align*}
    g(A) &= \sum_{n=0} \alpha_n A^n
    \intertext{Behauptung:}
    \sbr{A^n,B} &= CnA^{n-1}
    \intertext{Beweis durch vollständige Induktion:\newline $n=1$:}
    \sbr{A,B} &= C \\
              &= 1CA^0
    \intertext{$n\to n+1$:}
    \sbr{A^{n+1},B} &= A^{n+1}B - BA^{n+1} \\
                    &= A^nAB - A^nBA + A^nBA - BA^nA \\
                    &= A^n\sbr{A,B} + \sbr{A^n,B}A \\
                    &= A^nC + CnA^{n-1}A \\
                    &= (n+1)CA^n
    \intertext{Behauptung durch vollständiger Induktion Bewiesen. Daraus folgt}
    \sbr{g(A),B} &= \sum_{n=0} \alpha_n \sbr{A^n,B} \\
                 &= C\sum_{n=0} \alpha_n n A^{n-1} \\
                 &= C\dod{}{A} \sum_{n=0} \alpha_n A^n \\
                 &= C\dod{}{A} g(A) \\
                 &= Cg'(A).
\end{align*}

\section{Hermitesche Konjugation von Produkten von Operatoren}

\section{Auf- und Absteigeoperatoren}

\subsection{}

\subsection{}

Es soll gezeigt werden, das für den Besetzungszahloperator $\hat n =
\frac1{\hbar\omega_0}\hat a_+\hat a_-$ die Eigenwertgleichung $\hat nu_n(x) =
nu_n(x)$ gilt.
\begin{align*}
    \hat n u_n(x) &= \frac1{\hbar\omega_0}\hat a_+\hat a_- u_n(x) \\
                  &= \frac1{\hbar\omega_0}\hat a_+ \sqrt{\hbar\omega_0n}u_{n-1}(x) \\
                  &= \sqrt{\frac{n}{\hbar\omega_0}} \hat a_+ u_{n-1}(x) \\
                  &= \sqrt{\frac{n}{\hbar\omega_0}} \sqrt{\hbar\omega_0n}u_n(x) \\
                  &= nu_n(x)
\end{align*}
Die Eigenwertgleichung ist also erfüllt.

\end{document}
