% Für Seitenformatierung

\documentclass[DIV=15]{scrartcl}

% Zeilenumbrüche

\parindent 0pt
\parskip 6pt

% Für deutsche Buchstaben und Synthax

\usepackage[ngerman]{babel}

% Für Auflistung mit speziellen Aufzählungszeichen

\usepackage{paralist}

% zB für \del, \dif und andere Mathebefehle

\usepackage{amsmath}
\usepackage{commath}
\usepackage{amssymb}

% Für Literatur/bibliography

%\usepackage[backend=biber , style=alphabetic , hyperref=true]{biblatex}

% Für \SIunit[]{} und \num in deutschem Stil

\usepackage[output-decimal-marker={,}]{siunitx}
\DeclareSIUnit\clight{\ensuremath{c}}

% Schriftart und encoding

\usepackage[utf8]{inputenc}
% Bitstream charter als default
\usepackage[charter, greekuppercase=italicized]{mathdesign}
% Lato, als sans default
\renewcommand{\sfdefault}{fla}

% Für \sfrac{}{}, also inline-frac

\usepackage{xfrac}

% Für Einbinden von pdf-Grafiken

\usepackage{graphicx}

% Umfließen von Bildern

\usepackage{floatflt}

% Für weitere Farben

\usepackage{color}

% Für Streichen von z.B. $\rightarrow$

\usepackage{centernot}

% Für Befehl \cancel{}

\usepackage{cancel}

% Für Links nach außen und innerhalb des Dokumentes

\usepackage{hyperref}

% Für Layout von Links

\hypersetup{
	citecolor=black,
	colorlinks=true,
	linkcolor=black,
	urlcolor=blue,
}

% Verschiedene Mathematik-Hilfen

\newcommand \e[1]{\cdot10^{#1}}
\newcommand\p{\partial}

\newcommand\half{\frac 12}
\newcommand\shalf{\sfrac12}

\newcommand\skp[2]{\left\langle#1,#2\right\rangle}
\newcommand\mw[1]{\left\langle#1\right\rangle}
\newcommand \eexp[1]{\mathrm{e}^{#1}}
\newcommand \dexp[1]{\exp\del{#1}}
\newcommand \dsin[1]{\sin\del{#1}}
\newcommand \dcos[1]{\cos\del{#1}}
\newcommand \dtan[1]{\tan\del{#1}}
\newcommand \darccos[1]{\arcos\del{#1}}
\newcommand \darcsin[1]{\arcsin\del{#1}}
\newcommand \darctan[1]{\arctan\del{#1}}

% Nabla und Kombinationen von Nabla

\renewcommand\div[1]{\skp{\nabla}{#1}}
\newcommand\rot{\nabla\times}
\newcommand\grad[1]{\nabla#1}
\newcommand\laplace{\triangle}
\newcommand\dalambert{\mathop{{}\Box}\nolimits}

%Für komplexe Zahlen

\newcommand \ii{\mathrm i}
\renewcommand{\Im}{\mathop{{}\mathrm{Im}}\nolimits}
\renewcommand{\Re}{\mathop{{}\mathrm{Re}}\nolimits}

%Für Bra-Ket-Notation

\newcommand\bra[1]{\left\langle#1\right|}
\newcommand\ket[1]{\left|#1\right\rangle}
\newcommand\braket[2]{\left\langle#1\left.\vphantom{#1 #2}\right|#2\right\rangle}
\newcommand\braopket[3]{\left\langle#1\left.\vphantom{#1 #2 #3}\right|#2\left.\vphantom{#1 #2 #3}\right|#3\right\rangle}


\newcounter{thezettel}
\setcounter{thezettel}{3}
\renewcommand\thesection{\arabic{thezettel}.\arabic{section}}


\title{physik421 - Übung \arabic{thezettel}}
\author{Lino Lemmer \\ \small{l2@uni-bonn.de} \and Frederike Schrödel \and Simon Schlepphorst\\ \small{s2@uni-bonn.de}}

\begin{document}
\maketitle

\section{Fouriertransformation}

\subsection{Abbildung von Ableitungsoperator auf Multiplikation}

\subsection{}

\subsection{Abbildung von Produkt auf Faltung}

\subsection{Vertauschbarkeit von Ableitung und Faltung}

\section{Gaußintegrale}
Für diese Aufgabe ist eine Wahrscheinlichkeitsdichte von 
\[
    p(x) = C\ee^{-\frac 12ax^2=bx} 
\]
wobei $x$ eine Zufallsgröße ist und gilt: $a,b\in\Re$ und $a>0$.

\subsection{Identität}
Es soll gezeigt werden, dass gilt:
\[
    I = \int^\infty_{-\infty}\ee^{\frac{x^2}{2}}\dif x = \sqrt{2\pi}
\]
Den Tipp nutzend berechne ich zunächst $I^2$ und schreib dies dann mit $\vec x = r(\cos\phi,\sin\phi)$ und $\dif^2x = r\dif r\dif\phi$  in Polarkoordinaten um.
\begin{align*}
    I^2 &= \int^\infty_{-\infty}\dif x\ee^{-\frac{x^2}{2}}\int^\infty_{-\infty}\dif y\ee^{-\frac{y^2}{2}} \\
        &= \int_{\Re^2}\dif^2x\ee^{\frac{x^2}{2}} \\
        &= \int^{2\pi}_0\dif\phi\int^\infty_0\dif rr\ee^{-\frac{r^2}{2}} \\
\end{align*}
an dieser Stelle nutze ich folgenden Trick:
\begin{align*}
    \dod {\ee^{-\frac{r^2}{2}}}{r} &= \frac{-r}{2}\ee^{\frac{-r^2}{2}} \\
    \implies \int -\frac r2 \ee^{\frac{r^2}2} &= \ee^{\frac{r^2}2} \\
    \implies -2\int -\frac r2 \ee^{\frac{r^2}2} &= -2\ee^{\frac{r^2}2} \\
\end{align*}
Wenn ich damit weiter rechne erhalte ich als gesammt Lösung des Integrals:
\begin{align*}
    I^2 &= \int^{2\pi}_0\dif\phi\int^\infty_0\dif rr\ee^{-\frac{r^2}{2}} \\
        &=-2\pi\ee^{-\frac{r^2}2}|^\infty_{r=0} \\
        &= 2\pi \\
    I &= \sqrt{2\pi}
\end{align*}
Somit haben wir es gezeigt.

\subsection{Normierung}
Nach einer Normierung muss das Integral der Wahrscheinlcihkeitsdichte über den Raum eins ergeben.
\begin{align*}
    \int^\infty_{-\infty} P(x)\dif x &= int^\infty_{-\infty}\dif xC\ee^{-\frac 12ax^2+bx} \\
                                     &= 1 \\
\end{align*}
Zunächst formen wir den Exponenten um und substituieren.
\begin{align*}
    -\frac 12a^2+bx &= -\frac 12a(x^2-2\frac bax) \\
                    &= -\frac 12a(x^2-2\frac bax+\frac{b^2}{a^2}-\frac{b^2}{a^2}) \\
                    &= -\frac 12a[(x-\frac ba)^2-\frac{b^2}{a^2}] \\
\end{align*}
Mit $\xi := x-\frac ba$ erhalten wir $\ee^{1\frac 12 a(\xi^2-\frac{b^2}{a^2})}$.
Nun kann man über di Potenzgesetze die $\ee$-Funktion auseinander ziehen und den konstanten Term mit dem $C$ vor das Integral Ziehen.
\begin{align*}
    1&:= C\ee^{\frac{b^2}{2a}}\int^\infty_{-\infty}\dif\xi\ee^{-\frac 12a\xi^2} \qquad|\text{ mit }\psi^2 = \frac 12a\xi^2 \implies \psi=\sqrt{\frac a2}\xi \implies \dif\xi=\sqrt{\frac 2a}\dif \psi\\ 
     &= C\ee^{\frac{b^2}{2a}}\sqrt{\frac 2a}\int^\infty_{-\infty}\dif\psi\ee^{-\psi^2} \\
     &= C\ee^{\frac{b^2}{2a}}\sqrt{\frac 2a}\sqrt{\pi} \\
     \implies C &= \frac{1}{\ee^{\frac{b^2}{2a}}\sqrt{\frac 2a}\sqrt{\pi}} \\
\end{align*}

\subsection{Mittelwert}
Da es sich offensichtlich um eine Gaußverteilung handelt ist der Mittelwert gegeben durch den Mittelpunkt der Gaußverteilung.

\section{Freies Wellenpaket}

\subsection{Impulsverteilung}

\subsection{Wahrscheinlichkeitsdichte}

\subsection{Geschwindigkeit}

\subsection{Änderung der Ortsschwankung mit der Zeit}

\subsection{Normierungskonstante}

\section{$\delta$-Potenzial im Impulsraum}

\end{document}
