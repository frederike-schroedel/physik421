% Für Seitenformatierung

\documentclass[DIV=15]{scrartcl}

% Zeilenumbrüche

\parindent 0pt
\parskip 6pt

% Für deutsche Buchstaben und Synthax

\usepackage[ngerman]{babel}

% Für Auflistung mit speziellen Aufzählungszeichen

\usepackage{paralist}

% zB für \del, \dif und andere Mathebefehle

\usepackage{amsmath}
\usepackage{commath}
\usepackage{amssymb}

% Für Literatur/bibliography

%\usepackage[backend=biber , style=alphabetic , hyperref=true]{biblatex}

% Für \SIunit[]{} und \num in deutschem Stil

\usepackage[output-decimal-marker={,}]{siunitx}
\DeclareSIUnit\clight{\ensuremath{c}}

% Schriftart und encoding

\usepackage[utf8]{inputenc}
% Bitstream charter als default
\usepackage[charter, greekuppercase=italicized]{mathdesign}
% Lato, als sans default
\renewcommand{\sfdefault}{fla}

% Für \sfrac{}{}, also inline-frac

\usepackage{xfrac}

% Für Einbinden von pdf-Grafiken

\usepackage{graphicx}

% TikZ

\usepackage{tikz}
\usetikzlibrary{arrows}

% Umfließen von Bildern

\usepackage{floatflt}

% Für weitere Farben

\usepackage{color}

% Für Streichen von z.B. $\rightarrow$

\usepackage{centernot}

% Für Befehl \cancel{}

\usepackage{cancel}

% Für Links nach außen und innerhalb des Dokumentes

\usepackage{hyperref}

% Für Layout von Links

\hypersetup{
	citecolor=black,
	colorlinks=true,
	linkcolor=black,
	urlcolor=blue,
}

% Verschiedene Mathematik-Hilfen

\newcommand \e[1]{\cdot10^{#1}}
\newcommand\p{\partial}

\newcommand\half{\frac 12}
\newcommand\shalf{\sfrac12}

\newcommand\skp[2]{\left\langle#1,#2\right\rangle}
\newcommand\mw[1]{\left\langle#1\right\rangle}
\newcommand \eexp[1]{\mathrm{e}^{#1}}
\newcommand \dexp[1]{\exp\del{#1}}
\newcommand \dsin[1]{\sin\del{#1}}
\newcommand \dcos[1]{\cos\del{#1}}
\newcommand \dtan[1]{\tan\del{#1}}
\newcommand \darccos[1]{\arccos\del{#1}}
\newcommand \darcsin[1]{\arcsin\del{#1}}
\newcommand \darctan[1]{\arctan\del{#1}}

% Nabla und Kombinationen von Nabla

\renewcommand\div[1]{\skp{\nabla}{#1}}
\newcommand\rot{\nabla\times}
\newcommand\grad[1]{\nabla#1}
\newcommand\laplace{\triangle}
\newcommand\dalambert{\mathop{{}\Box}\nolimits}

%Für komplexe Zahlen

\newcommand \ee{\mathrm e}
\newcommand \ii{\mathrm i}
\renewcommand{\Im}{\mathop{{}\mathrm{Im}}\nolimits}
\renewcommand{\Re}{\mathop{{}\mathrm{Re}}\nolimits}

%Für Bra-Ket-Notation

\newcommand\bra[1]{\left\langle#1\right|}
\newcommand\ket[1]{\left|#1\right\rangle}
\newcommand\braket[2]{\left\langle#1\left.\vphantom{#1 #2}\right|#2\right\rangle}
\newcommand\braopket[3]{\left\langle#1\left.\vphantom{#1 #2 #3}\right|#2\left.\vphantom{#1 #2 #3}\right|#3\right\rangle}
