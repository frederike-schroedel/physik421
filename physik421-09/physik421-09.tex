% Für Seitenformatierung

\documentclass[DIV=15]{scrartcl}

% Zeilenumbrüche

\parindent 0pt
\parskip 6pt

% Für deutsche Buchstaben und Synthax

\usepackage[ngerman]{babel}

% Für Auflistung mit speziellen Aufzählungszeichen

\usepackage{paralist}

% zB für \del, \dif und andere Mathebefehle

\usepackage{amsmath}
\usepackage{commath}
\usepackage{amssymb}

% Für Literatur/bibliography

%\usepackage[backend=biber , style=alphabetic , hyperref=true]{biblatex}

% Für \SIunit[]{} und \num in deutschem Stil

\usepackage[output-decimal-marker={,}]{siunitx}
\DeclareSIUnit\clight{\ensuremath{c}}

% Schriftart und encoding

\usepackage[utf8]{inputenc}
% Bitstream charter als default
\usepackage[charter, greekuppercase=italicized]{mathdesign}
% Lato, als sans default
\renewcommand{\sfdefault}{fla}

% Für \sfrac{}{}, also inline-frac

\usepackage{xfrac}

% Für Einbinden von pdf-Grafiken

\usepackage{graphicx}

% Umfließen von Bildern

\usepackage{floatflt}

% Für weitere Farben

\usepackage{color}

% Für Streichen von z.B. $\rightarrow$

\usepackage{centernot}

% Für Befehl \cancel{}

\usepackage{cancel}

% Für Links nach außen und innerhalb des Dokumentes

\usepackage{hyperref}

% Für Layout von Links

\hypersetup{
	citecolor=black,
	colorlinks=true,
	linkcolor=black,
	urlcolor=blue,
}

% Verschiedene Mathematik-Hilfen

\newcommand \e[1]{\cdot10^{#1}}
\newcommand\p{\partial}

\newcommand\half{\frac 12}
\newcommand\shalf{\sfrac12}

\newcommand\skp[2]{\left\langle#1,#2\right\rangle}
\newcommand\mw[1]{\left\langle#1\right\rangle}
\newcommand \eexp[1]{\mathrm{e}^{#1}}
\newcommand \dexp[1]{\exp\del{#1}}
\newcommand \dsin[1]{\sin\del{#1}}
\newcommand \dcos[1]{\cos\del{#1}}
\newcommand \dtan[1]{\tan\del{#1}}
\newcommand \darccos[1]{\arcos\del{#1}}
\newcommand \darcsin[1]{\arcsin\del{#1}}
\newcommand \darctan[1]{\arctan\del{#1}}

% Nabla und Kombinationen von Nabla

\renewcommand\div[1]{\skp{\nabla}{#1}}
\newcommand\rot{\nabla\times}
\newcommand\grad[1]{\nabla#1}
\newcommand\laplace{\triangle}
\newcommand\dalambert{\mathop{{}\Box}\nolimits}

%Für komplexe Zahlen

\newcommand \ii{\mathrm i}
\renewcommand{\Im}{\mathop{{}\mathrm{Im}}\nolimits}
\renewcommand{\Re}{\mathop{{}\mathrm{Re}}\nolimits}

%Für Bra-Ket-Notation

\newcommand\bra[1]{\left\langle#1\right|}
\newcommand\ket[1]{\left|#1\right\rangle}
\newcommand\braket[2]{\left\langle#1\left.\vphantom{#1 #2}\right|#2\right\rangle}
\newcommand\braopket[3]{\left\langle#1\left.\vphantom{#1 #2 #3}\right|#2\left.\vphantom{#1 #2 #3}\right|#3\right\rangle}


\newcounter{thezettel}
\setcounter{thezettel}{9}
\renewcommand\thesection{\arabic{thezettel}.\arabic{section}}



\title{physik421 - Übung \arabic{thezettel}}
\author{Lino Lemmer \\ \small{l2@uni-bonn.de} \and Frederike Schrödel \and Simon Schlepphorst\\ \small{s2@uni-bonn.de}}

\begin{document}
\maketitle

\section{Drehmatrix}

\subsection{}

Eine Drehmatrix erfüllt folgende Eigenschaften:
\begin{itemize}
    \item
        $\det(\mat{M}) = 1$
    \item
        $\mathbf{M}\cdot\mathbf{M}^\text{T} = \mathbb{1}$
\end{itemize}

Da für die gegebene Matrix jedoch gilt
\[
    \det(\mat D) = \frac1{\sqrt{2}},
\]
in den folgenden Teilaufgaben jedoch davon ausgegangen wird, die Matrix sei eine Drehmatrix, gehe ich von folgender Matrix aus:
\[
    \mat D = \frac{1}{\sqrt{2}}
    \begin{pmatrix}
        -1 & 0 & -1 \\
        0 & \sqrt2 & 0 \\
        1 & 0 & -1
    \end{pmatrix}.
\]
Für diese prüfe ich die oben genannten Eigenschaften.
\begin{align*}
    \det (\mat D) &= \del{\frac1{\sqrt{2}}}^3 \cdot
    \begin{vmatrix}
        -1 & 0 & -1 \\
        0 & \sqrt2 & 0 \\
        1 & 0 & -1
    \end{vmatrix} \\
    &= \del{\frac1{\sqrt{2}}}^3 \cdot\del{ - 0 \cdot 
    \begin{vmatrix}
        0 & 0 \\
        1 & -1
    \end{vmatrix} + \sqrt2 \cdot \begin{vmatrix}
        -1 & -1 \\
        1 & -1
    \end{vmatrix} - 0 \cdot \begin{vmatrix}
        -1 & -1 \\
        0 & 0
    \end{vmatrix}} \\
    &= \del{\frac1{\sqrt{2}}}^2 \cdot
    \begin{vmatrix}
    -1 & -1 \\
    1 & -1
    \end{vmatrix} \\
    &= \half \del{(-1)\cdot(-1)-1\cdot(-1)} \\
    &= 1
    \intertext{%
        Damit wäre die erste Bedingung erfüllt.
    }
    \mat D \cdot \mat D^\text{T} &= \half
    \begin{pmatrix}
        -1 & 0 & -1 \\
        0 & \sqrt2 & 0 \\
        1 & 0 & -1
    \end{pmatrix} \cdot
    \begin{pmatrix}
        -1 & 0 & 1 \\
        0 & \sqrt2 & 0 \\
        -1 & 0 & -1
    \end{pmatrix} \\
    &= \half
    \begin{pmatrix}
        1+1 & 0 & 1-1 \\
        0 & 2 & 0 \\
        1-1 & 0 1+1
    \end{pmatrix} \\
    &= \mathbb 1
    \intertext{%
        Beide Bedingungen sind erfüllt, es handelt sich also tatsächlich um eine Drehmatrix. Um nun herauszufinden, was sie für eine Drehung verursacht, lassen wir sie auf einen Vektor $\vec x$ wirken.
    }
    \mat D\cdot \vec x &= \frac 1{\sqrt2}
    \begin{pmatrix}
        -1 & 0 & -1 \\
        0 & \sqrt2 & 0 \\
        1 & 0 & -1
    \end{pmatrix} \cdot
    \begin{pmatrix}
        x_1 \\
        x_2 \\
        x_3
    \end{pmatrix}  \\
    &= \begin{pmatrix}
    \frac1{\sqrt2}\del{-x_1-x_3} \\
    x_2 \\
    \frac1{\sqrt2}\del{x_1-x_3} \mantag \label{eq:gedreht}
\end{pmatrix}.
\end{align*}
Da die zweite Komponente des Vektors durch die Drehung nicht verändert wird, handelt es sich ganz offensichtlich um eine Drehung um die $y$-Achse.

\subsection{}

Mit \eqref{eq:gedreht} erhalten wir für die beiden Vektoren
\begin{align*}
    \vec a' &= \mat D \cdot \del{0,-2,1}^\text{T} \\
            &= \begin{pmatrix}
    \frac{1}{\sqrt{2}}\cdot\del{-0 - 1} \\
    -2 \\
    \frac{1}{\sqrt{2}}\cdot\del{0 - 1}
    \end{pmatrix} \\
    &= \begin{pmatrix}
    -\frac1{\sqrt2} \\
    -2 \\
    -\frac1{\sqrt2}
\end{pmatrix} \\
    \vec b' &= \mat D \cdot \del{3,5,-4}^\text{T} \\
            &= \begin{pmatrix}
    \frac1{\sqrt2}\cdot\del{-3+4} \\
        5 \\
        \frac1{\sqrt{2}}\del{3+4}
    \end{pmatrix} \\
    &= \begin{pmatrix}
    \frac1{\sqrt2} \\
    5 \\
    \frac7{\sqrt2}
\end{pmatrix}
\end{align*}

\subsection{}

Da gilt 
\begin{align*}
    \skp{\vec a}{\vec b} &= \abs a \cdot \abs b \dcos{\alpha},
    \intertext{%
        mit dem Winkel zwischen den Vektoren $\alpha$, und sowohl die Beträge der Vektoren, als auch der Winkel zwischen ihnen bei einer Drehung konstant bleiben, sollte das Skalarprodukt vor und nach der Drehung gleich sein. Dies Prüfen wir nun.
    }
    \skp{\vec a}{\vec b} &= \skp{\begin{pmatrix}
    0 \\
    -2 \\
    1
    \end{pmatrix}}{\begin{pmatrix}
        3 \\
        5 \\
        -4
    \end{pmatrix}} \\
    &= 0 - 10 -4 \\
    &= -14 \\
    \skp{\vec a'}{\vec b'} &= \skp{\begin{pmatrix}
    -\frac1{\sqrt2} \\
    -2 \\
    -\frac1{\sqrt2}
    \end{pmatrix}}{\begin{pmatrix}
    \frac1{\sqrt2} \\
    5 \\
    \frac7{\sqrt2}
    \end{pmatrix}} \\
    &= -\half - 10 - \frac72 \\
    &= -14
\end{align*}
Unsere Vermutung ist somit bestätigt.


\end{document}
