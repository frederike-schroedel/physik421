%Autor: Jan Weber
%
\IfFileExists{header.tex}{% Für Seitenformatierung

\documentclass[DIV=15]{scrartcl}

% Zeilenumbrüche

\parindent 0pt
\parskip 6pt

% Für deutsche Buchstaben und Synthax

\usepackage[ngerman]{babel}

% Für Auflistung mit speziellen Aufzählungszeichen

\usepackage{paralist}

% zB für \del, \dif und andere Mathebefehle

\usepackage{amsmath}
\usepackage{commath}
\usepackage{amssymb}

% Für Literatur/bibliography

%\usepackage[backend=biber , style=alphabetic , hyperref=true]{biblatex}

% Für \SIunit[]{} und \num in deutschem Stil

\usepackage[output-decimal-marker={,}]{siunitx}
\DeclareSIUnit\clight{\ensuremath{c}}

% Schriftart und encoding

\usepackage[utf8]{inputenc}
% Bitstream charter als default
\usepackage[charter, greekuppercase=italicized]{mathdesign}
% Lato, als sans default
\renewcommand{\sfdefault}{fla}

% Für \sfrac{}{}, also inline-frac

\usepackage{xfrac}

% Für Einbinden von pdf-Grafiken

\usepackage{graphicx}

% Umfließen von Bildern

\usepackage{floatflt}

% Für weitere Farben

\usepackage{color}

% Für Streichen von z.B. $\rightarrow$

\usepackage{centernot}

% Für Befehl \cancel{}

\usepackage{cancel}

% Für Links nach außen und innerhalb des Dokumentes

\usepackage{hyperref}

% Für Layout von Links

\hypersetup{
	citecolor=black,
	colorlinks=true,
	linkcolor=black,
	urlcolor=blue,
}

% Verschiedene Mathematik-Hilfen

\newcommand \e[1]{\cdot10^{#1}}
\newcommand\p{\partial}

\newcommand\half{\frac 12}
\newcommand\shalf{\sfrac12}

\newcommand\skp[2]{\left\langle#1,#2\right\rangle}
\newcommand\mw[1]{\left\langle#1\right\rangle}
\newcommand \eexp[1]{\mathrm{e}^{#1}}
\newcommand \dexp[1]{\exp\del{#1}}
\newcommand \dsin[1]{\sin\del{#1}}
\newcommand \dcos[1]{\cos\del{#1}}
\newcommand \dtan[1]{\tan\del{#1}}
\newcommand \darccos[1]{\arcos\del{#1}}
\newcommand \darcsin[1]{\arcsin\del{#1}}
\newcommand \darctan[1]{\arctan\del{#1}}

% Nabla und Kombinationen von Nabla

\renewcommand\div[1]{\skp{\nabla}{#1}}
\newcommand\rot{\nabla\times}
\newcommand\grad[1]{\nabla#1}
\newcommand\laplace{\triangle}
\newcommand\dalambert{\mathop{{}\Box}\nolimits}

%Für komplexe Zahlen

\newcommand \ii{\mathrm i}
\renewcommand{\Im}{\mathop{{}\mathrm{Im}}\nolimits}
\renewcommand{\Re}{\mathop{{}\mathrm{Re}}\nolimits}

%Für Bra-Ket-Notation

\newcommand\bra[1]{\left\langle#1\right|}
\newcommand\ket[1]{\left|#1\right\rangle}
\newcommand\braket[2]{\left\langle#1\left.\vphantom{#1 #2}\right|#2\right\rangle}
\newcommand\braopket[3]{\left\langle#1\left.\vphantom{#1 #2 #3}\right|#2\left.\vphantom{#1 #2 #3}\right|#3\right\rangle}
}{% Für Seitenformatierung

\documentclass[DIV=15]{scrartcl}

% Zeilenumbrüche

\parindent 0pt
\parskip 6pt

% Für deutsche Buchstaben und Synthax

\usepackage[ngerman]{babel}

% Für Auflistung mit speziellen Aufzählungszeichen

\usepackage{paralist}

% zB für \del, \dif und andere Mathebefehle

\usepackage{amsmath}
\usepackage{commath}
\usepackage{amssymb}

% Für Literatur/bibliography

%\usepackage[backend=biber , style=alphabetic , hyperref=true]{biblatex}

% Für \SIunit[]{} und \num in deutschem Stil

\usepackage[output-decimal-marker={,}]{siunitx}
\DeclareSIUnit\clight{\ensuremath{c}}

% Schriftart und encoding

\usepackage[utf8]{inputenc}
% Bitstream charter als default
\usepackage[charter, greekuppercase=italicized]{mathdesign}
% Lato, als sans default
\renewcommand{\sfdefault}{fla}

% Für \sfrac{}{}, also inline-frac

\usepackage{xfrac}

% Für Einbinden von pdf-Grafiken

\usepackage{graphicx}

% Umfließen von Bildern

\usepackage{floatflt}

% Für weitere Farben

\usepackage{color}

% Für Streichen von z.B. $\rightarrow$

\usepackage{centernot}

% Für Befehl \cancel{}

\usepackage{cancel}

% Für Links nach außen und innerhalb des Dokumentes

\usepackage{hyperref}

% Für Layout von Links

\hypersetup{
	citecolor=black,
	colorlinks=true,
	linkcolor=black,
	urlcolor=blue,
}

% Verschiedene Mathematik-Hilfen

\newcommand \e[1]{\cdot10^{#1}}
\newcommand\p{\partial}

\newcommand\half{\frac 12}
\newcommand\shalf{\sfrac12}

\newcommand\skp[2]{\left\langle#1,#2\right\rangle}
\newcommand\mw[1]{\left\langle#1\right\rangle}
\newcommand \eexp[1]{\mathrm{e}^{#1}}
\newcommand \dexp[1]{\exp\del{#1}}
\newcommand \dsin[1]{\sin\del{#1}}
\newcommand \dcos[1]{\cos\del{#1}}
\newcommand \dtan[1]{\tan\del{#1}}
\newcommand \darccos[1]{\arcos\del{#1}}
\newcommand \darcsin[1]{\arcsin\del{#1}}
\newcommand \darctan[1]{\arctan\del{#1}}

% Nabla und Kombinationen von Nabla

\renewcommand\div[1]{\skp{\nabla}{#1}}
\newcommand\rot{\nabla\times}
\newcommand\grad[1]{\nabla#1}
\newcommand\laplace{\triangle}
\newcommand\dalambert{\mathop{{}\Box}\nolimits}

%Für komplexe Zahlen

\newcommand \ii{\mathrm i}
\renewcommand{\Im}{\mathop{{}\mathrm{Im}}\nolimits}
\renewcommand{\Re}{\mathop{{}\mathrm{Re}}\nolimits}

%Für Bra-Ket-Notation

\newcommand\bra[1]{\left\langle#1\right|}
\newcommand\ket[1]{\left|#1\right\rangle}
\newcommand\braket[2]{\left\langle#1\left.\vphantom{#1 #2}\right|#2\right\rangle}
\newcommand\braopket[3]{\left\langle#1\left.\vphantom{#1 #2 #3}\right|#2\left.\vphantom{#1 #2 #3}\right|#3\right\rangle}
}

\usepackage{makeidx}
\makeindex

\usepackage{tabularx}
\newcolumntype{L}[1]{>{\raggedright\arraybackslash}p{#1}} % linksbündig mit Breitenangabe
\newcolumntype{C}[1]{>{\centering\arraybackslash}p{#1}} % zentriert mit Breitenangabe
\newcolumntype{R}[1]{>{\raggedleft\arraybackslash}p{#1}} % rechtsbündig mit Breitenangabe

\author{Jan Weber\\\small{s6jawebe@uni-bonn.de}\and Tobias Brauell\\\small{s6tobrau@uni-bonn.de}}
\title{Theoretische Physik III}
\subtitle{Übungsblatt 2}
\usepackage{tikz}
%\usepackage{tikz-3dplot}
\usetikzlibrary{calc}
\usetikzlibrary{decorations.markings}

\usepackage{paralist}

\usepackage{todonotes}
\usepackage{listings}

\usepackage{mathtools}

\usepackage{cancel}

\begin{document}
\maketitle

\section{Entartung von Energieniveaus}
Zu zeigen ist, dass bei einer eindimensionalen, stationären Schrödingergleichung mit einem zeitunabhängigen Potential $V(x)$ kein Energieniveau des diskreten Spektrums entartet ist.
\begin{align}
  i\hbar\partial_t \psi(x,t)=\left(-\frac{\hbar^2}{2m}\Delta^2+V(x)\right)\psi(x,t)
\end{align}
Für zwei Wellenfunktionen gilt also:
\begin{align}
  H\psi_1&=E\psi_1		H\psi_2&&=E\psi_2		\\
  E\psi_1&=\left(-\frac{\hbar^2}{2m}\nabla^2+V(x)\right)\psi_1		E\psi_2&&=\left(-\frac{\hbar^2}{2m}\nabla^2+V(x)\right)\psi_2	\\
  E\psi_1\psi_2&=\left(\left(-\frac{\hbar^2}{2m}\nabla^2+V(x)\right)\psi_1\right)\psi_2		E\psi_2\psi_1&&=\left(\left(-\frac{\hbar^2}{2m}\nabla^2+V(x)\right)\psi_2\right)\psi_1
\end{align}
Beide Gleichungen werden nun gleich gesetzt und die Klammern aufgelöst:
\begin{align}
  \left(\left(-\frac{\hbar^2}{2m}\nabla^2+V(x)\right)\psi_1\right)\psi_2&=\left(\left(-\frac{\hbar^2}{2m}\nabla^2+V(x)\right)\psi_2\right)\psi_1	\\
  \cancel{-\frac{\hbar^2}{2m}}\psi''_1\psi_2+\cancel{V(x)\psi_1\psi_2}&=\cancel{-\frac{\hbar^2}{2m}}\psi_1\psi''_2+\cancel{V(x)\psi_1\psi_2}
\end{align}
Das liefert also:
\begin{align}
  \psi''_1\psi_2 - \psi_1\psi''_2 = 0
\end{align}
Nun Intergrieren wir zwei mal partiell:
\begin{align}
  \int \left(\psi''_1\psi_2 - \psi_1\psi''_2\right) dx &= \psi'_1\psi_2 - \cancel{\int \left(\psi'_1\psi'_2\right)dx} - \psi_1\psi'_2 + \cancel{\int \left(\psi'_1\psi'_2\right)dx} + C \\
  &= \psi'_1\psi_2 - \psi_1\psi'_2
\end{align}

\todo[inline]{Hier muss es noch weiter gehen.}



\section{Delta-Potential}
In dieser Aufgabe sollen Eigenwerte und normierte Eigenfunktionen eines $\delta$-Potentials $V(x)=-\alpha\delta(x)$ berechnet werden. Dabei sollen die Energien als negativ betrachtet werden. Es ergibt sich also die folgende Schrödingergleichung:
\begin{align}
  i\hbar\partial_t \psi(x,t)=\left(-\frac{\hbar^2}{2m}\nabla^2-\alpha\delta(x)\right)\psi(x,t)&=-E\psi(x,t)	\\
  -\frac{\hbar^2}{2m}\psi''-\alpha\delta(x)\psi&=-E\psi
\end{align}
Für die Grenzbedingung bei $x=0$:
\begin{align}
  E \int^\epsilon_{-\epsilon}dx \psi &= \int^\epsilon_{-\epsilon}dx \frac{\hbar^2}{2m} \psi'' + \int^\epsilon_{-\epsilon}dx \alpha\delta(x)\psi		\\
  &= \frac{\hbar^2}{2m}\left(\psi'(\epsilon)-\psi'(-\epsilon)\right)+\alpha\psi(0)
\end{align}
Nun wird der Grenzwert $\epsilon\rightarrow 0$ gebildet:
\begin{align}
  \lim\limits_{\epsilon \rightarrow 0} \left(\frac{\hbar^2}{2m}\left(\psi'(\epsilon)-\psi'(-\epsilon)\right)+\alpha\psi(0)\right)	\\
  \Rightarrow \psi'(0^+)-\psi'(0^-)=-\frac{2m\alpha}{\hbar}\psi(0)
\end{align}
Damit die Grenzwerte $0^+$ und $0^-$ existieren muss die Wellengleichung $\psi$ an der Stelle $\psi(0)$ stetig sein.

\todo[inline]{Hier muss es noch weiter gehen.}



\section{Stückweise konstantes Potential}




\section{Streuung am Potentialwall}







\end{document}