% Für Seitenformatierung

\documentclass[DIV=15]{scrartcl}

% Zeilenumbrüche

\parindent 0pt
\parskip 6pt

% Für deutsche Buchstaben und Synthax

\usepackage[ngerman]{babel}

% Für Auflistung mit speziellen Aufzählungszeichen

\usepackage{paralist}

% zB für \del, \dif und andere Mathebefehle

\usepackage{amsmath}
\usepackage{commath}
\usepackage{amssymb}

% Für Literatur/bibliography

%\usepackage[backend=biber , style=alphabetic , hyperref=true]{biblatex}

% Für \SIunit[]{} und \num in deutschem Stil

\usepackage[output-decimal-marker={,}]{siunitx}
\DeclareSIUnit\clight{\ensuremath{c}}

% Schriftart und encoding

\usepackage[utf8]{inputenc}
% Bitstream charter als default
\usepackage[charter, greekuppercase=italicized]{mathdesign}
% Lato, als sans default
\renewcommand{\sfdefault}{fla}

% Für \sfrac{}{}, also inline-frac

\usepackage{xfrac}

% Für Einbinden von pdf-Grafiken

\usepackage{graphicx}

% Umfließen von Bildern

\usepackage{floatflt}

% Für weitere Farben

\usepackage{color}

% Für Streichen von z.B. $\rightarrow$

\usepackage{centernot}

% Für Befehl \cancel{}

\usepackage{cancel}

% Für Links nach außen und innerhalb des Dokumentes

\usepackage{hyperref}

% Für Layout von Links

\hypersetup{
	citecolor=black,
	colorlinks=true,
	linkcolor=black,
	urlcolor=blue,
}

% Verschiedene Mathematik-Hilfen

\newcommand \e[1]{\cdot10^{#1}}
\newcommand\p{\partial}

\newcommand\half{\frac 12}
\newcommand\shalf{\sfrac12}

\newcommand\skp[2]{\left\langle#1,#2\right\rangle}
\newcommand\mw[1]{\left\langle#1\right\rangle}
\newcommand \eexp[1]{\mathrm{e}^{#1}}
\newcommand \dexp[1]{\exp\del{#1}}
\newcommand \dsin[1]{\sin\del{#1}}
\newcommand \dcos[1]{\cos\del{#1}}
\newcommand \dtan[1]{\tan\del{#1}}
\newcommand \darccos[1]{\arcos\del{#1}}
\newcommand \darcsin[1]{\arcsin\del{#1}}
\newcommand \darctan[1]{\arctan\del{#1}}

% Nabla und Kombinationen von Nabla

\renewcommand\div[1]{\skp{\nabla}{#1}}
\newcommand\rot{\nabla\times}
\newcommand\grad[1]{\nabla#1}
\newcommand\laplace{\triangle}
\newcommand\dalambert{\mathop{{}\Box}\nolimits}

%Für komplexe Zahlen

\newcommand \ii{\mathrm i}
\renewcommand{\Im}{\mathop{{}\mathrm{Im}}\nolimits}
\renewcommand{\Re}{\mathop{{}\mathrm{Re}}\nolimits}

%Für Bra-Ket-Notation

\newcommand\bra[1]{\left\langle#1\right|}
\newcommand\ket[1]{\left|#1\right\rangle}
\newcommand\braket[2]{\left\langle#1\left.\vphantom{#1 #2}\right|#2\right\rangle}
\newcommand\braopket[3]{\left\langle#1\left.\vphantom{#1 #2 #3}\right|#2\left.\vphantom{#1 #2 #3}\right|#3\right\rangle}


\newcounter{thezettel}
\setcounter{thezettel}{2}
\renewcommand\thesection{\arabic{thezettel}.\arabic{section}}

\title{physik421 - Übung \arabic{thezettel}}
\author{Lino Lemmer \\ \small{l2@uni-bonn.de} \and Frederike Schrödel \and Simon Schlepphorst\\ 
 \small{s2@uni-bonn.de}}

\begin{document}
\maketitle

\section{Entartung von Energieniveaus}

\section{$\delta$-Potenzial}

\section{Stückweise konstantes Potenzial}

\subsection{Anschlussbedingungen}

Die Schrödinger-Gleichung lautet
\begin{align*}
    \ii\hbar\dpd{}{t} \psi(q,t) &= \del{-\frac{\hbar^2}{2m}\dpd[2]{}{q} + V(q)} \psi(q,t),
    \intertext{%
        mit
    }
    V(q) &=
    \begin{cases}
        V_1 & ,-\infty<q<-q_0 \text{ (Bereich 1)}\\
        0 & ,-q_0<q<+q_0 \text{ (Bereich 2)} \\
        V_3 & ,+q_0<q<+\infty \text{ (Bereich 3)}
    \end{cases}.
    \intertext{%
        Jetzt machen wir den Faktorisierungsansatz $\psi(q,t) = u(q)v(t)$:
    }
    \frac{\ii\hbar\dpd{v}{t}}{v(t)} &= \frac{-\frac{\hbar^2}{2m}\dpd[2]{u}{q}}{u(q)} + V(q) \overset{!}{=} E,
    \intertext{%
        wobei $E$ eine konstante ist. Für die zeitabhängige Lösung ergibt sich
    }
    v(t) &= \eexp{-\ii\frac{E}{\hbar}t}.
    \intertext{%
        Nun betrachten wir die ortsabhängige Funktion.
    }
    \frac{\hbar^2}{2m} \dpd[2]{u}{q} &= \del{V(q) - E} u(q) \\
    \iff \qquad \dpd[2]{u}{q} &= \frac{2m}{\hbar^2}\del{V(q) - E}u
    \intertext{%
        Für Bereich 1 ergibt sich mit $\kappa_1^2=\frac{2m}{\hbar^2}\del{V_1 -E}$ die Differentialgleichung
    }
    \dpd[2]{u_1}{q} &= \kappa_1^2u_1,
    \intertext{%
        mit der Lösung
    }
    u_1(q) &= a_1\eexp{\kappa_1 q} + a_2\eexp{-\kappa_1 q}.
    \intertext{%
        Bereich 3 lässt sich mit $\kappa_3^2=\frac{2m}{\hbar^2}\del{V_3 -E}$ analog lösen. Wir erhalten
    }
    u_3(q) &= c_1\eexp{\kappa_3 q} + c_2\eexp{-\kappa_3 q}.
    \intertext{%
        Für Bereich 2 erhalten wir mit $k^2=\frac{2m}{\hbar^2}E$ die Differentialgleichung
    }
    \dpd[2]{u_2}{q} &= -k^2u_2,
    \intertext{%
        mit der Lösung
    }
    u_2(q) &= b_1\eexp{\ii kq} + b_2\eexp{-\ii kq}.
    \intertext{%
        Da sowohl die Funktion, als auch die Ableitung stetig sein muss erhalten wir die Randbedingungen
    }
    u_1\del{-q_0} &= u_2\del{-q_0} \\
    u_1'\del{-q_0} &= u_2'\del{-q_0} \\
    u_2\del{q_0} &= u_3\del{q_0} \\
    u_2'\del{q_0} &= u_3'\del{q_0},
    \intertext{%
        oder ausgeschrieben
    }
    a_1\eexp{-\kappa_1 q_0} + a_2 \eexp{\kappa_1 q_0} &= b_1 \eexp{-\ii kq_0} + b_2 \eexp{\ii kq_0} \\
    \kappa_1\del{a_1\eexp{-\kappa_1 q_0} - a_2\eexp{\kappa_1 q_0}} &= \ii k\del{ b_1 \eexp{-\ii kq_0} - b_2 \eexp{\ii kq_0}} \\
    b_1 \eexp{\ii kq_0} + b_2 \eexp{-\ii kq_0} &= c_1\eexp{\kappa_3 q_0} + c_2 \eexp{-\kappa_3 q_0} \\
    \ii k\del{ b_1 \eexp{\ii kq_0} - b_2 \eexp{-\ii kq_0}} &= \kappa_3\del{c_1\eexp{\kappa_3 q_0} - c_2\eexp{-\kappa_3 q_0}} \\
\end{align*}

\subsection{Eigenenergien und transzendente Gleichung}

\subsection{Bestimmungsgleichung}

\subsection{Diskretes Eigenwertspektrum}

\subsection{Eigenwerte bei unterschiedlichen Potenzialen}

\section{Streuung am Potenzialwall}

\subsection{Form der Wellenfunktion}

\subsection{Anschlussbedingungen}

\subsection{Reflexions- und Transmissionskoeffizient}

\end{document}
