% Für Seitenformatierung

\documentclass[DIV=15]{scrartcl}

% Zeilenumbrüche

\parindent 0pt
\parskip 6pt

% Für deutsche Buchstaben und Synthax

\usepackage[ngerman]{babel}

% Für Auflistung mit speziellen Aufzählungszeichen

\usepackage{paralist}

% zB für \del, \dif und andere Mathebefehle

\usepackage{amsmath}
\usepackage{commath}
\usepackage{amssymb}

% Für Literatur/bibliography

%\usepackage[backend=biber , style=alphabetic , hyperref=true]{biblatex}

% Für \SIunit[]{} und \num in deutschem Stil

\usepackage[output-decimal-marker={,}]{siunitx}
\DeclareSIUnit\clight{\ensuremath{c}}

% Schriftart und encoding

\usepackage[utf8]{inputenc}
% Bitstream charter als default
\usepackage[charter, greekuppercase=italicized]{mathdesign}
% Lato, als sans default
\renewcommand{\sfdefault}{fla}

% Für \sfrac{}{}, also inline-frac

\usepackage{xfrac}

% Für Einbinden von pdf-Grafiken

\usepackage{graphicx}

% Umfließen von Bildern

\usepackage{floatflt}

% Für weitere Farben

\usepackage{color}

% Für Streichen von z.B. $\rightarrow$

\usepackage{centernot}

% Für Befehl \cancel{}

\usepackage{cancel}

% Für Links nach außen und innerhalb des Dokumentes

\usepackage{hyperref}

% Für Layout von Links

\hypersetup{
	citecolor=black,
	colorlinks=true,
	linkcolor=black,
	urlcolor=blue,
}

% Verschiedene Mathematik-Hilfen

\newcommand \e[1]{\cdot10^{#1}}
\newcommand\p{\partial}

\newcommand\half{\frac 12}
\newcommand\shalf{\sfrac12}

\newcommand\skp[2]{\left\langle#1,#2\right\rangle}
\newcommand\mw[1]{\left\langle#1\right\rangle}
\newcommand \eexp[1]{\mathrm{e}^{#1}}
\newcommand \dexp[1]{\exp\del{#1}}
\newcommand \dsin[1]{\sin\del{#1}}
\newcommand \dcos[1]{\cos\del{#1}}
\newcommand \dtan[1]{\tan\del{#1}}
\newcommand \darccos[1]{\arcos\del{#1}}
\newcommand \darcsin[1]{\arcsin\del{#1}}
\newcommand \darctan[1]{\arctan\del{#1}}

% Nabla und Kombinationen von Nabla

\renewcommand\div[1]{\skp{\nabla}{#1}}
\newcommand\rot{\nabla\times}
\newcommand\grad[1]{\nabla#1}
\newcommand\laplace{\triangle}
\newcommand\dalambert{\mathop{{}\Box}\nolimits}

%Für komplexe Zahlen

\newcommand \ii{\mathrm i}
\renewcommand{\Im}{\mathop{{}\mathrm{Im}}\nolimits}
\renewcommand{\Re}{\mathop{{}\mathrm{Re}}\nolimits}

%Für Bra-Ket-Notation

\newcommand\bra[1]{\left\langle#1\right|}
\newcommand\ket[1]{\left|#1\right\rangle}
\newcommand\braket[2]{\left\langle#1\left.\vphantom{#1 #2}\right|#2\right\rangle}
\newcommand\braopket[3]{\left\langle#1\left.\vphantom{#1 #2 #3}\right|#2\left.\vphantom{#1 #2 #3}\right|#3\right\rangle}


\newcounter{thezettel}
\setcounter{thezettel}{8}
\renewcommand\thesection{\arabic{thezettel}.\arabic{section}}

\newcommand{\ui}[1]{\int_{-\infty}^{\infty}\dif {#1}\;}
\newcommand\ccancel[2][black]{\renewcommand\CancelColor{\color{#1}}\cancel{#2}}


\title{physik421 - Übung \arabic{thezettel}}
\author{Lino Lemmer \\ \small{l2@uni-bonn.de} \and Frederike Schrödel \and Simon Schlepphorst\\ \small{s2@uni-bonn.de}}

\begin{document}
\maketitle

\section{Eigenvektoren in einer Orthonormalbasis}
\subsection{}
Es soll bestimmt werden, ob $\sigma_y$ hermitesch ist.
\begin{align*}
    \sigma^\dagger_y &= 
    \begin{pmatrix}
        0 &\ii \\
        -\ii &0
    \end{pmatrix}^* \\
    &= \begin{pmatrix}
        0 &-\ii \\
        \ii &0
    \end{pmatrix} \\
    &= \sigma_y \implies \text{hermitesch}
\end{align*}

Außerdem sollen die Eigenwerte und Eigenvektoren bestimmt werden.
\begin{align*}
    \abs{\sigma_y-\lambda\mathbb{1}} &= (-\lambda)^2-(-\ii)\ii \\
                                    &= \lambda^2-1 \\
                                    &\implies \lambda_1 =1, \qquad \lambda_2 = -1
    \lambda_1 \\
    &\implies \underbrace{\frac{1}{\sqrt{2}}}_\text{normierung}\begin{pmatrix}
        1 \\ \ii
    \end{pmatrix} = \Phi_1\\
    \lambda_2 &\implies \frac 1{\sqrt 2}\begin{pmatrix}
        1 \\ -\ii
    \end{pmatrix} = \Phi_2
\end{align*}

\subsection{}
Es soll Orthogonalität und Vollständigkeit der Eigenvektoren überprüft werden.

Orthogonalität: $\Phi_1*\Phi_2 = 1-1 = 0$

Vollständigkeit: $\tilde{\eta}_1 = \frac 1{\sqrt 2}(\Phi_1 +\Phi_2) \qquad \tilde{\eta}_2 = \frac 1{\sqrt 2}(\Phi_1 -\Phi_2) $

Da man die Basis durch die Eigenvektoren darstellen kann, müssen die Eigenvektoren vollständig sein.

\subsection{}
\begin{align*}
    P_\ii &= \Phi_\ii\dagger\Phi_\ii \\
    P_1 &= 
    \begin{pmatrix}
        1 \\ \ii
    \end{pmatrix}\begin{pmatrix}
        1 & -\ii
    \end{pmatrix} 
    = \frac 12
    \begin{pmatrix}
        1 &-\ii \\
        \ii &1
    \end{pmatrix} \\
     P_2 &= 
    \begin{pmatrix}
        1 \\ -\ii
    \end{pmatrix}\begin{pmatrix}
        1 & \ii
    \end{pmatrix} 
    = \frac 12
    \begin{pmatrix}
        1 & \ii \\
        -\ii & 1
    \end{pmatrix} \\
    P_1\Phi_1 &= \frac 1{2\sqrt 2}
    \begin{pmatrix}
        1 & -\ii \\
        \ii & 1
    \end{pmatrix} \begin{pmatrix}
        1 \\ \ii 
    \end{pmatrix}
    = \frac 1{\cancel{2}\sqrt 2}\begin{pmatrix}
        \cancel{2} \\ \cancel{2} \ii
    \end{pmatrix} 
    = \Phi_1 \\ 
    P_1\Phi_2 &= \frac 1{2\sqrt 2}
    \begin{pmatrix}
        1 & -\ii \\
        \ii & 1
    \end{pmatrix} \begin{pmatrix}
        1 \\ -\ii 
    \end{pmatrix}
    = \begin{pmatrix}
        0 \\ 0
    \end{pmatrix} \\
    P_2\Phi_1 &= \frac 1{2\sqrt 2}
    \begin{pmatrix}
        1 & \ii \\
        -\ii & 1
    \end{pmatrix} \begin{pmatrix}
        1 \\ \ii 
    \end{pmatrix}
    = \begin{pmatrix}
        0 \\ 0
    \end{pmatrix} \\
    P_2\Phi_2 &= \frac 1{2\sqrt 2}
    \begin{pmatrix}
        1 & \ii \\
        -\ii & 1
    \end{pmatrix} \begin{pmatrix}
        1 \\ -\ii 
    \end{pmatrix}
    = \frac 1{\cancel{2}\sqrt 2}\begin{pmatrix}
        \cancel{2} \\ \cancel{2} \ii
    \end{pmatrix} 
    =\Phi_2\\ 
\end{align*}

\subsection{}
Zu zeigen $P_iP_j = \delta_{ij}$

\begin{align*}
    P_1P_1 &= 
    \frac 12 \begin{pmatrix} 1 &-\ii \\ \ii & 1 \end{pmatrix} 
        \frac 12 \begin{pmatrix} 1 & -\ii \\ \ii & 1 \end{pmatrix} \\
                                   &= \frac 14 \begin{pmatrix} 2 & -2i \\ 2i & 2 \end{pmatrix} \\
                                   &=\frac 12 \begin{pmatrix} 1 &-\ii \\ -\ii & 1 \end{pmatrix} \\
                                   &=P_1 \\
    &...
\end{align*}
Zu zeigen: $\sum_\ii P_\ii = \mathbb{1}$
\begin{align*}
    P_1+P_2 &= \frac 12 \begin{pmatrix} 1 &-\ii \\ \ii & 1 \end{pmatrix} +
        \frac 12 \begin{pmatrix} 1 & \ii \\ -\ii & 1 \end{pmatrix} \\
                                   &= \frac 12 \begin{pmatrix} 2 & 0 \\ 0 & 2 \end{pmatrix} \\
                                   &=\frac 12 \begin{pmatrix} 1 & 0 \\ 0 & 1 \end{pmatrix} \\
                                   &= \mathbb{1}
\end{align*}

\section{Vollständiger Satz von Operatoren}

In einen dreidimensinalen Raum sind zwei Operatoren durch
\[
    H = \hbar\omega 
    \begin{pmatrix}
        1 & 0 & 0 \\
        0 & 0 & 0 \\
        0 & 0 & 1
    \end{pmatrix}
\]
und
\[
    B = b
    \begin{pmatrix}
        0 & 0 & 1 \\
        0 & 1 & 0 \\
        1 & 0 & 0 
    \end{pmatrix}
\]

\subsection{}
$H$ ist hermitsch, da man es transponieren und komplex konjugieren kann ohne dass es sich ändert. Bei $B$ müsste dafür gelten, dass $b$ reell ist.

\subsection{}
Um zu zeigen, dass $H$ und $B$ vertauschen muss gezeigt werden, dasss $[H,B] = 0$ gilt.
\begin{align*}
    [H,B] &= HB-BH \\
          &= \hbar\omega b
    \begin{pmatrix}
        1 &0 &0 \\
        0 &0 &0 \\
        0 &0 &1 \\
    \end{pmatrix}\begin{pmatrix}
        0 &0 &1 \\
        0 &1 &0 \\
        1 &0 &0 \\
    \end{pmatrix} 
    -\hbar\omega b
   \begin{pmatrix}
        0 &0 &1 \\
        0 &1 &0 \\
        1 &0 &0 \\
    \end{pmatrix} \begin{pmatrix}
        1 &0 &0 \\
        0 &0 &0 \\
        0 &0 &1 \\
    \end{pmatrix} \\
    &= \hbar\omega b
    \begin{pmatrix}
        0 &0 &1 \\
        0 &0 &0 \\
        1 &0 &0 \\
    \end{pmatrix} 
    -\hbar\omega b
    \begin{pmatrix}
        0 &0 &1 \\
        0 &0 &0 \\
        1 &0 &0 \\
    \end{pmatrix} \\
    &= 0
\end{align*}

\subsection{}
Als nächstes sollen drei Vektoren bestimmt werden, die sowohl von $H$ als auch von $B$ Eigenvektoren sind.
Zunächst bestimme ich die Eigenwerte der beiden Matrizen um daraus die Eigenräume zu bestimmen, aus denen ich gemeinsame Vektoren ermittele.
Die Eigenwerte ermittel ich mit $\text{det}(A-\lambda \mathbb{1})$. 
Damit erhalte ich als Eigenwerte zu der Matrix von $H$ $\lambda_{1,2}=\hbar\omega$ und $\lambda_3=0$.
Für die Matrix von $B$ erhalte ich ebenfalls einen doppelten Eigenwert $\lambda_{1,2}=b$ und einen einzelnen $\lambda_3=-b$.
Nun bestimme ich die dazu gehörigen Eigenvektoren.

$H$ hat zu den Eigenwert $\hbar\omega$ die Eigenvektoren $(1,0,0)^T$ und $(0,0,1)^T$ und zu den EW den EV $(0,1,0)^T$. Zu $B$ erhalte ich für $b$ die EV $(0,1,0^T$ und $(1,0,1)^T$ und für $-b$ $(1,0,-1)^T$.
Damit erhält man als gemeinsame EV $(0,1,0)^T$, $(1,0,1)^T$ und $(1,0,-1)^T$.
EW: $(0,b)$ , $(\hbar,b)$ und $(\hbar\omega,-b)$.

\subsection{}
Die EW $\lambda_{H,i}$ reichen nicht aus um den Zustand $i$ eindeutig zu bestimmen. Jeder Zustand $a(1,0,0,)^T+b(0,0,1)^T$ mit $a,b \in \mathbb R $ gehört zum EW $\hbar\omega$. Hingegen wissen wir, das zum EW $0$ $i=(0,1,0)^T$ sein muss.

Die Paaren von EW $(\lambda_{H,i},\lambda_{B,i})$ reichen aus um den Zustand $i$ eindeutig zu bestimmen. Ist nun z.B. $\lambda_{H,i} =\lambda_\omega$ so wissen wir, weil es sich um einen EV von $B$ handelt, das entweder $i=(1,0,1)^T$ oder $i=(1,0,-1)^T$. Diese haben verschiedene EW von $B$: $b$ und $-b$. Beide EW zusammen reichen somit aus um $i$festzulegen. $\implies {H,B}$ vollständiger Satz.

\subsection{}
\[
    B'=b\begin{pmatrix}
        0 &0 &0 \\
        0 &1 &0 \\
        0 &0 &0
    \end{pmatrix}
\]
Hierfür erhalte ich die EW $0$ (doppelt) und $b$ und daraus die EV $(1,0,0)^T$, $(0,0,1)^T$ und für $b$ $(0,1,0)^T$
Somit haben $B'$ und $H$ degenerierten Eigenraum (von $(1,0,0,^T$ und $(0,0,1)$ aufgespannt).
Deswegen wird der Zustand durch diese EV auch nicht eindeutig estgelegt. $\implies$ kein vollständiger Satz.

\section{Heisenbergdarstellung von Operatoren}
\subsection{}
Zu zeigen:
$\sbr{A^n,B} = nA^{n-1}[A,B]$ falls $[A,B] = c \in \mathbb C$
Induktionsanfang: $n=1$
\[
    [A,B] = 1A^0[A,B]
\]
Induktionsschritt: angenommen die Behauptung gilt für n: zu zeigen $\sbr{A^{n+1},B} = (n+1)A^n[A,B]$
\begin{align*}
    \implies \sbr{A^{n+1},B} &= A^{n+1}B-BA^{n+1} \\
                         &= A^{n+1}B-A^nBA+A^nBA-BA^{n+1} \\
                         &= A^n[A,B]+\sbr{A^n,B}A \\
                         &= cA^n+nA^{n-1}cA \\
                         &= (n+1)A^nc \\
                         &= (n+1)A^n[A,B]
\end{align*}

\subsection{}
Zu zeigen $[H,p] \in\mathbb C$ mit $H = p^2/2m-qE_x$:
\begin{align*}
    [H,p] &= \sbr{\frac{p^2}{2m}-qE_x,p} \\
          &= -qE\underbrace{\sbr{x,p}}_{\ii\hbar} \qquad|\text{ da }\sbr{p,p}=0 \\
          &= -\ii\hbar qE \in\mathbb C
\end{align*}

\subsection{}
Zu zeigen: $\hat p_H(t) = \hat p_H(0)+qE+\hat e_x$ mit $H=\hat p^2/2m-qE_x$
\begin{align*}
    \hat p_H(t) &= \ee^{\ii Ht/\hbar}\hat{p}\ee^{-\ii Ht/\hbar} \\
                 &= \sum^\infty_{n=0}\frac 1{n!}\del{\frac \ii\hbar}^nt^nH^n\hat p\ee^{\ii Ht/\hbar} \\
                 &= \sum^\infty_{n=0} \frac 1{(n-0)!}\del{\frac \ii\hbar t}^{n-1}H^{n-1}\del{\frac \ii\hbar t}(-\ii\hbar qE\hat{e_x})+\hat p \\
                 &= \hat p + qE+\hat e_x \\
                 &=\hat p_H(0)+qE+\hat e_x
\end{align*}


    



\end{document}
